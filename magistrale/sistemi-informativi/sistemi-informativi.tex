\documentclass[a4paper, 11pt]{article}
\usepackage[english]{babel}
\usepackage[utf8]{inputenc}
\usepackage{amsmath}
\usepackage{graphicx}
\usepackage{float}
\usepackage{fixltx2e}
\usepackage{listings}
\usepackage{color}
\usepackage{latexsym}
\usepackage{lstautogobble}
\usepackage[colorinlistoftodos]{todonotes}
\usepackage[margin=3cm]{geometry}
\usepackage{hyperref}
\usepackage{libertine}
\usepackage{tikz}
\hypersetup{
	hidelinks, 
	colorlinks = true,
	linkcolor = black,
}

\usetikzlibrary{shapes, arrows}

\newtheorem{definit}{Definizione}[subsection]
\newcommand{\chainto}{$ \to $}

\begin{document}
	\clearpage
	\begin{titlepage}
		\centering
		\vspace*{\fill}
		{\scshape\LARGE Università degli Studi di Verona \par}
		\vspace{1.5cm}
		\line(1,0){280} \\
		{\huge\bfseries Sistemi informativi\par}
		\line(1,0){280} \\
		\vspace{0.5cm}
		{\scshape\Large Riassunto dei principali argomenti\par}
		\vspace{2cm}
		{\Large\itshape Davide Bianchi\par}
		\vspace{1cm}
		
		\vspace{5cm}
		\vspace*{\fill}
		% Bottom of the page
		{\large \today\par}
	\end{titlepage}
	\thispagestyle{empty}
	\newpage
	\tableofcontents
	\newpage
	
	
	\section{Teoria dell'organizzazione}
	\subsection{Introduzione}
	Iniziamo con alcune definizioni \textit{estremamente} tediose.
	\begin{definit}[Sistema informativo]
		Il Sistema Informativo (SI) è la componente (sottosistema) di una organizzazione che gestisce le informazioni di interesse.
	\end{definit}
	
	\begin{definit}[Organizzazione]
		Un’organizzazione è : \begin{itemize}
			\item il processo attraverso il quale tale insieme di persone viene strutturato secondo i principi di divisione del lavoro e coordinamento;
			\item il risultato del processo di divisione del lavoro e coordinamento.
		\end{itemize}
	\end{definit}
	
	\begin{definit}[Azienda]
		Un'azienda, nell'economia aziendale, è un'organizzazione di uomini e mezzi finalizzata alla soddisfazione di bisogni umani attraverso la produzione, la distribuzione o il consumo di beni economici.
	\end{definit}
	
	\begin{definit}[Organizzazione aziendale]
		Il processo attraverso il quale l'insieme di persone che partecipano direttamente allo svolgimento dell'attività dell'azienda viene strutturato secondo i principi di divisione del lavoro e coordinamento.
	\end{definit}
	L'organizzazione aziendale ha sempre almeno i macro processi operativo e gestionale, e dispone di risorse materiali, umane e informative.
	
	\begin{definit}[Tecnologie informatiche]
		Insieme di sistemi, strumenti e tecniche predisposti per automatizzare il trattamento delle informazioni.
	\end{definit}
	
	Un sistema informativo aziendale è una collezione di elementi interconnessi che gesticono la raccolta, l'elaborazione e la restituzione di informazioni. \\
	
	Un sistema produttivo aziendale è basato su \textit{obiettivi} (output atteso), \textit{input} ed \textit{output} effettivi. Definiamo inoltre i concetti di \textit{efficienza}, ovvero il costo di raggiungimento degli obiettivi, e di \textit{efficacia}, ovvero il grado di raggiungimento degli obiettivi. A grandi linee: \[ \text{\textit{Efficienza}} = \frac{Output}{Input} \qquad \text{\textit{Efficacia}} = \frac{Output}{Obiettivi}  \]
	
	Efficienza ed efficacia subiscono impatti differenti rispetto ad una innovazione delle risorse tecnologiche. Nel caso dell'efficienza: \begin{itemize}
		\item Riduzione dei costi unitari;
		\item Aumento di produzione a parità di risorse;
		\item Incentivo alla crescita delle dimensioni organizzative;
		\item Maggiore complessità strutturale;
		\item Cambiamenti della struttura organizzativa.
	\end{itemize}
	
	Nel caso dell'efficacia: \begin{itemize}
		\item Più efficiente uso dei fattori produttivi a parità di volumi di produzione (economie di scopo);
		\item Razionalizzazione dell'uso di risorse;
		\item Maggiore efficienza spesso legata a differenziazione dei prodotti e all’ampliamento della gamma.
	\end{itemize}
	
	L'organizzazione è un sistema aperto, influenzato da variabili ambientali, che vengono riassunte nell'\textit{incertezza ambientale}.
	L'incertezza ambientale determina i requisiti di capacità elaborativa della organizzazioni e l'adeguatezza del sistema informativo.
	
	\begin{definit}[Capacità elaborativa]
		Adeguatezza di un’organizzazione rispetto alle necessità di elaborare informazioni a essa imposte dai propri obiettivi e dal contesto in cui opera.
	\end{definit}
	
	L'ambiente è riassunto nel modello della piramide di Anthony, ovvero una piramide divisa in livelli gerarchici.
	Il layer più alto si occupa delle decisioni strategiche, quello centrale delle decisioni direzionali e quello basso di quelle operative.
	% FIGURA DELLA PIRAMIDE DI ANTHONY
	
	
	
	\subsection{Informazione come risorsa organizzativa}
	\paragraph{Caratteri principali.} L'informazione è la vera risorsa nelle attività organizzative, infatti viene scambiata ed elaborata; è immateriale, non è facilmente divisibile, può essere soggetta ad obsolescenza e si autorigenera.
	
	La capacità autorigenerativa dell'informazione permette di instaurare circoli virtuosi di generatori di conoscenza e di arricchimento delle informazioni disponibili, che si traducono in un incremento dei processi produttivi.
	
	\paragraph{Overload e underload informativo.} L'\textit{overload informativo} è un aumento incontrollato dell'informazione disponibile, che eccede la capacità di elaborazione individuale, con conseguente rallentamento nell'elaborazione. L'\textit{underload informativo} è invece una disponibilità di informazione al di sotto delle capacità individuali, con conseguente presa di decisioni in tempi brevi.
	
	\subsection{Sistemi informativi verticali e orizzontali}
	I sistemi informativi possono essere immaginati a due versi. 
	
	I sistemi informativi verticali sono stati i primi ad essere supportati dai sistemi informatici, tuttavia al crescere dell'incertezza i vertici sono sovraccaricati dai compiti decisionali. 
	
	I sistemi informativi orizzontali invece sono costruiti sulla delega delle decisioni e sui collegamenti sullo stesso layer che aumentano la capacità elaborativa (team di lavoro, task force).
	
	\section{Classificazione dei sistemi informativi}
	Vi sono varie possibili classificazioni dei sistemi informativi, ovvero: \begin{itemize}
		\item tipologie di SI disposti lungo la piramide aziendale (definizioni e funzioni attribuite a seconda del loro livello nella piramide);
		\item tipologie di SI diposti nelle varie aree gestionali dell'impresa.
	\end{itemize}

	\subsection{SI disposti lungo la piramide aziendale}
	I SI disposti lungo la piramide di Anthony sono i seguenti:
	\begin{enumerate}
		\item \textit{Transaction Processing Systems (TPS)}: gestione delle transazioni, quali ordini ecc. Sono alla base della piramide;
		
		\item \textit{Management Information Systems (MIS)}: sono al livello immediatamente sopra ai TPS e rappresentano periodicamente le informazioni raccolte dai TPS. Sono alla base del sistema di reportistica delle aziende;
		
		\item \textit{Decision Support Systems (DSS)}: affiancano il management delle decisioni di non routine e permettono di simulare ipotesi per verificare la validità di una gestione.
		
		\item \textit{Executive Information Systems (EIS)}: sono al vertice della gerarchia, aiutano i senior manager alla gestione.
	\end{enumerate}
	
	\subsection{Portafoglio applicativo}
	Il portafoglio applicativo è l'insieme delle applicazioni utili in azienda. È diviso in 3 segmenti principali: \begin{itemize}
		\item Portafoglio direzionale: insieme delle applicazioni informatiche a supporto dei cicli di pianificazione strategica;
		
		\item Portafoglio istituzionale: applicazioni informatiche per i processi di supporto all’amministrazione;
		
		\item Portafoglio operativo: applicazioni informatiche per i processi primari dell’azienda.
	\end{itemize}

	Il portafoglio istituzionale è un'area con elevate potenzialità di informatizzazione, a causa dei grandi volumi di dati che li coinvolgono e la forte proceduralità. Come conseguenza si hanno riduzioni nei tempi e nei costi di elaborazione, inoltre la pianificazione risulta più efficace.
	
	Il portafoglio operativo contiene le applicazioni informatiche necessarie ai procedimenti coinvolti nella catena del valore di Porter, ossia: \\
	
	Gestione materie prime \chainto Trasformazione \chainto Vendita \chainto Distribuzione \chainto Postvendita \\
	
	ovvero la catena di azioni finalizzate a produrre valore per il cliente. Il portafoglio applicativo ovviamente è specifico di ogni settore industriale, e comporta un aumento della complessità gestibile nei processi aziendali, permettendo inoltre la sincronizzazione dei dati in un'azienda (basi di dati condivise).
	
	Il portafoglio applicativo è andato informatizzandosi col passare degli anni, a cominciare dalle procedure per automatizzare attività singole, ai pacchetti MRP (Manufacturing Resource Planning), che contenevano i primi database e pacchetti integrati, ai CIM (Computer Integrated Manifacturing), che automatizzano interi segmenti produttivi, ai sistemi ERP (Enterprise Resource Planning) che consentono di gestire ogni fase produttiva e sfruttano architetture client-server e pacchetti integrati con un unico modello dati.
	
	Negli ultimi anni sono andati sviluppandosi i sistemi CRM (Customer Relationship Management), che forniscono interi cicli di assistenza al cliente, gestioni avanzate di distribuzione, vendita e postvendita. Negli anni 2000 si è sviluppato l'E-Procurement, ovvero l'informatizzazione del buy-side delle imprese, e utilizza pacchetti per l'intero ciclo di acquisto e architetture basate su tecnologie web.
	
	\section{Ingegneria Sociale}
	Definizione di Ingegneria Sociale a caldo:\begin{definit}[Ingegneria Sociale]
		Manipolazione della naturale tendenza alla fiducia dell’essere umano, architettata e realizzata dall’ingegnere sociale con l’obiettivo ottenere informazioni che permettano libero accesso e informazioni di valore del sistema.
	\end{definit}

	La figura dell'ingegnere sociale mira a stabilire confidenza con la vittima, sviluppando ogni possibile scenario di difficoltà e preparandosi ad evaderlo. Prima di tutto ciò viene la fase di \textit{footprinting}, ovvero di raccolta delle informazioni, l'analisi dell'azienda, dei suoi sistemi di comunicazione, della posta, ecc. I primi ingegneri sociali furono i \textit{phreaker}, che utilizzavano la rete telefonica sfruttando i sistemi e i dipendenti dell'azienda per arrivare a dati sensibili.
	
	La falla da sfruttare è quindi data da operatori umani, che spesso gestiscono le informazioni sensibili ma ignorano le procedure di sicurezza, magari non sono nemmeno consapevoli delle informazioni che stanno gestendo e che dovrebbero custodire. 
	Ovviamente le vittime perfette per un attacco di ingegneria soociale sono le persone che non hanno nulla da perdere nel fornire informazioni sensibili, che sottostimano il valore delle informazioni, sottostimano le procedure di sicurezza oppure che non valutano le conseguenze delle proprie azioni.
	
	\subsection{Attacco tipico di Ingegneria Sociale}
	\paragraph{Fasi di un attacco di SE.} Un generico attacco di SE si svolge nel seguente modo: \begin{itemize}
		\item una fase fisica di raccolta di informazioni attraverso persone, documenti e luoghi;
		\item una fase psicologica di impersonificazione e persuasione del personale adatto ad essere una tipica vittima
	\end{itemize}

	\paragraph{Fase fisica.} Gli strumenti essenziali alla fase fisica sono gli strumenti di comunicazione più disparati. L'obiettivo di questa fase sono password, server e router, e si possono raggiungere tramite una giusta combo di uso della tecnologia (phishing, lancio di malware, ...) e interazione col personale (truffe telefoniche, dumpster diving, rovistare negli hdd dismessi...).
	
	\paragraph{Fase psicologica.} È necessario fare leva sulla fiducia che una persona è per inclinazione disposta a concedere, facendo leva sui bisogni primari dell'uomo (fisiologici, di sicurezza, ecc.), secondo la gerarchia di Maslow. 
	
	Gli attacchi di social engeneering sfruttano quindi le debolezze della persona singola, ossia la disponibilità e la fede che una persona è disposta ad affidare ad un possibile attaccante.
	
	\paragraph{Phishing.} \begin{definit}[Phishing]
		Tecnica di ingegneria sociale basata sul principio della supposta autorità che utilizza un messaggio di posta elettronica per acquisire informazioni personali riservate (password, dati finanziari, numero di carta di credito) con la finalità del furto di identità.
	\end{definit}

	Il phishing è basato sul concetto di \textit{mail spoofing}, ossia sull'inviare mail a nomi di terzi, per il semplice motivo che la persona che manda la mail non è autenticata dal server di posta elettronica.
	
	
	
	
	
	
	
	
	
	
\end{document}