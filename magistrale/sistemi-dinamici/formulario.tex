\documentclass[a4paper]{article}
\usepackage[italian]{babel}
\usepackage{fetamont}
\usepackage[T1]{fontenc}
\usepackage[utf8]{inputenc}
\usepackage{physics}
\usepackage{mathtools, nccmath}
\usepackage{amsmath}
\usepackage{amsfonts}
\usepackage{amsthm}
\usepackage{amssymb}
\usepackage{bm}
\usepackage{empheq}
\usepackage{mathtools}
\usepackage{mathdots}
\usepackage{enumitem}
\usepackage{multicol}
\usepackage{lipsum}
\usepackage{tikz}
\usetikzlibrary{shapes,arrows,positioning,calc}

\usepackage[thinlines,thiklines]{easybmat}
\usepackage{fullpage}

\usepackage[top=1cm, bottom=.25cm, left=0.7cm, right=0.7cm]{geometry}

\usepackage{fancyhdr}
\pagestyle{fancy}
\rhead{\nouppercase{\textbf{Formulario}}}
\chead{}
\lfoot{}
\cfoot{\thepage}
\rfoot{}
\renewcommand{\headrulewidth}{0pt}


\renewcommand{\vec}{\bm}
\pagenumbering{gobble}

\usepackage[OT1]{eulervm}

\begin{document}
	\section*{Controllabilità}
	\[
		\text{\textbf{Gramiano di Controllabilità :} }\ \vec{\Gamma}(t) = \int_{0}^{t} e^{\vec{A}\tau} \vec{B}\vec{B}^T e^{\vec{A}^T\tau}d\tau\quad
		\text{ non singolare } (\det \neq 0)
	\]
	\[
		\text{\textbf{Matrice di Controllabilità :} }\
		\mathcal{T} = 
		\begin{bmatrix}
		\vec{B} & \vec{A}\vec{B} & \vec{A}^2\vec{B} & \cdots & \vec{A}^{n-1}\vec{B}
		\end{bmatrix}\quad
		\rank (\mathcal{T}) = n
	\]
	\begin{multicols}{2}
		\begin{align*}
			\mathcal{T}^{-1} &= 
			\begin{bmatrix}
				\alpha_1 & \alpha_2 & \alpha_3 & \cdots & \alpha_{n-1} & 1 \\
				\alpha_2 & \alpha_3 & \vdots & \iddots & 1 & 0 \\
				\alpha_3 & \vdots & \alpha_{n-1} & \iddots & 0 & 0 \\
				\vdots & \alpha_{n-1} & 1 & \cdots & 0 & 0 \\
				\alpha_{n-1} & 1 & 0 & \cdots & 0 & 0 \\
				1 & 0 & 0 & \cdots & 0 & 0
			\end{bmatrix} \\ \\
			\vec{A}_c &=
			\begin{bmatrix}
				0 & 1 & 0 & \cdots & 0 \\
				0 & 0 & 1 & \cdots & 0 \\
				\vdots & \vdots & \vdots & \ddots & \vdots \\
				0 & 0 & 0 & \cdots & 1 \\
				\dfrac{-\alpha_0}{\alpha_n} & \dfrac{-\alpha_1}{\alpha_n} & \dfrac{-\alpha_2}{\alpha_n} & \cdots & \dfrac{-\alpha_{n-1}}{\alpha_n}
			\end{bmatrix}
			\begin{bmatrix}
				0 \\
				0 \\
				0 \\
				\vdots\\
				0 \\
				1
			\end{bmatrix}
			= \vec{B}_c \\
			\vec{C}_c &=
			\begin{bmatrix}
				\beta_0 & \beta_1 & \beta_2 & \beta_3 & \beta_4 & \cdots & \beta_{n-1} 
			\end{bmatrix}
			\begin{bmatrix}
				\beta_n
			\end{bmatrix}
			= \vec{D}_c
		\end{align*}
		\columnbreak
		
		\[
			\vec{P} = \mathcal{T}^{-1} \mathcal{T}
		\]
		\[
			\vec{A}_c = \vec{P}^{-1}\vec{A}\vec{P}\quad	\vec{B}_c = \vec{P}^{-1}\vec{B}
		\]
		\[
			\vec{C}_c = \vec{C}\vec{P}\quad \vec{D}_c = \vec{D}
		\]
		\begin{align*}
			W(s) &= \dfrac{\beta_m s^m + \beta_{m-1}s^{m-1} + \dots + \beta_1 s + \beta_0 }{\alpha_n s^n + \alpha_{n-1} s^{n-1} + \dots + \alpha_1 s + \alpha_0 } \\ 
				&= \vec{C}(s\mathbb{I} - \vec{A})^{-1}\vec{B} + \vec{D} \\ \\
			P(s) &= \det (s\mathbb{I} - \vec{A})\quad \rightarrow \ \alpha \\
			P_c (s) &= (s-\overline{\lambda}_1)\cdots (s-\overline{\lambda}_n) \quad \rightarrow \ \overline{\alpha} \\ \\
			\vec{K}_c &= 
			\begin{bmatrix}
				\overline{\alpha}_0 - \alpha_0 & \overline{\alpha}_1 - \alpha_1 & \cdots & \overline{\alpha}_{n-1} - \alpha_{n-1}
			\end{bmatrix} \\
			\vec{K} &= \vec{K}_c\vec{P}^{-1}
		\end{align*}
		
	\end{multicols}
	\vspace{-.5cm}
	\noindent\hrulefill
	\vspace{-.5cm}
	\section*{Osservabilità}
	\[
		\text{\textbf{Gramiano di Osservabilità :} }\ \vec{O}(t) = \int_0^t e^{\vec{A}^T\tau}\vec{C}^T\vec{C}e^{\vec{A}\tau}d\tau\quad
		\text{ non singolare } (\det \neq 0)
	\]
	\[
	\text{\textbf{Matrice di Osservabilità :} }\
		\mathcal{O} =
		\begin{bmatrix}
			\vec{C} &\vec{C}\vec{A} & \vec{C}\vec{A}^2 & \cdots & \vec{C}\vec{A}^{n-1} 
		\end{bmatrix}^T \quad
		\rank (\mathcal{O}) = n
	\]
	\begin{multicols}{2}
		\begin{align*}
			\mathcal{O}^{-1} &= \mathcal{T}^{-1} \\
			\vec{A}_o &=
			\begin{bmatrix}
				0 & 0 & \cdots & 0 & -\alpha_0/\alpha_n \\
				1 & 0 & \cdots & 0 & -\alpha_1/\alpha_n \\
				0 & 1 & \cdots & 0 & -\alpha_2/\alpha_n \\
				\vdots & \vdots & \ddots & \vdots & \vdots \\
				0 & 0 & \cdots & 1 & -\alpha_{n-1}/\alpha_n
			\end{bmatrix}
			\begin{bmatrix}
				\beta_0 \\
				\beta_1 \\
				\beta_2 \\
				\vdots \\
				\beta_{n-1}
			\end{bmatrix}
			= \vec{B}_o \\
			\vec{C}_o &=
			\begin{bmatrix}
				0 & 0 & 0 & 0 & 0 & \cdots & 0 & 1  
			\end{bmatrix}
			\begin{bmatrix}
				\beta_n
			\end{bmatrix}
			= \vec{D}_o
		\end{align*}
		\columnbreak
		
		\textbf{Osservatore di Luenberger:}
		\[
			\vec{P}_o = \mathcal{O}^{-1} \mathcal{O} \qquad
			\vec{L}_o =
			\begin{bmatrix}
				\overline{\alpha}_0 - \alpha_0 \\ 
				\overline{\alpha}_1 - \alpha_1 \\
				\vdots \\
				\overline{\alpha}_{n-1} - \alpha_{n-1}
			\end{bmatrix}
		\]
		\[
			\vec{A}_o = \vec{A} - \vec{L}_o\vec{C} \quad
			\vec{B}_o =
			\begin{bmatrix}
				\vec{B} & \vec{L}_o
			\end{bmatrix}\quad
			\vec{C}_o = \vec{C}\quad \vec{D}_o = \vec{D}
		\]
		\[
			\vec{L} = \vec{P}_o^{-1}\vec{L}_o \quad
			\overline{\vec{u}} (t) = 
			\begin{bmatrix}
				\vec{u}(t) \\
				\vec{y}(t)
			\end{bmatrix}
		\]
		
	\end{multicols}
	\vspace{-.5cm}
	\noindent\hrulefill
	\vspace{-.5cm}
	\section*{Sylvester per Matrice di transizione di stato}
	\[
		e^{\vec{A}t} = \sum_{i = 0}^{n - 1} \beta_i (t)\vec{A}^i = 
		\beta_0 (t)\vec{I} + \beta_1(t)\vec{A} + \cdots + \beta_{n-1}(t)\vec{A}^{n-1}
	\]
	\vspace{-1cm}
	\begin{multicols}{2}
		$ \mu_a = 1 $
		\vspace{-.5cm}
		\[
			\begin{cases}
				\beta_0 (t) + \lambda_1\beta_1 (t) + \lambda_1^2\beta_2 (t) + \cdots + \lambda_1^{n-1} \beta_{n-1}(t) = e^{\lambda_1 t} \\
				\beta_0 (t) + \lambda_2\beta_1 (t) + \lambda_2^2\beta_2 (t) + \cdots + \lambda_2^{n-1} \beta_{n-1}(t) = e^{\lambda_2 t} \\
				\qquad \vdots \qquad \vdots \\
				\beta_0 (t) + \lambda_n\beta_1 (t) + \lambda_n^2\beta_2 (t) + \cdots + \lambda_n^{n-1} \beta_{n-1}(t) = e^{\lambda_n t} \\
			\end{cases}
		\]
		\columnbreak
		
		$ \mu_a > 1 $
		\vspace{-.2cm}
		\begin{empheq}[left={\empheqlbrace}]{alignat*=2}
			\beta_0 (t) + \lambda\beta_1 (t) + \cdots + \lambda^{n-1} \beta_{n-1} (t) &= e^{\lambda t} \\
			\dfrac{d}{d\lambda}\left( \beta_0(t) + \lambda\beta_1 (t) + \cdots + \lambda^{n-1}\beta_{n-1}(t) \right) &= \dfrac{d}{d\lambda}e^{\lambda t} \\
			\qquad \vdots \qquad \qquad \qquad \qquad & \qquad \vdots \\
			\dfrac{d^{\nu -1}}{d\lambda^{\nu -1}}\left( \beta_0(t) + \lambda\beta_1 (t) + \cdots + \lambda^{n-1}\beta_{n-1}(t) \right) &= \dfrac{d^{\nu -1}}{d\lambda^{\nu -1}}e^{\lambda t}
		\end{empheq}
	\end{multicols}
	\noindent
	\textbf{Autovalori complessi } $ \lambda, \lambda' = \alpha \pm j \omega $:
	\begin{empheq}[left={\empheqlbrace}]{alignat*=2}
		\beta_0 (t) + Re(\lambda)\beta_1 (t) + Re(\lambda^2)\beta_2 (t) + \cdots & + Re(\lambda^{n-1}) \beta_{n-1}(t) &= e^{\lambda_1 t} cos(\omega t)\\
		Im(\lambda)\beta_1 (t) + Im(\lambda )^2\beta_2 (t) + \cdots & + Im(\lambda^{n-1}) \beta_{n-1}(t) &= e^{\lambda t} sin(\omega t)
	\end{empheq}
	dove $ \ Re(\lambda) = \alpha\ $ e $ \ Im(\lambda) = \omega $.

	\section*{Varie}
	\subsection*{Funzione di trasferimento}
	Dato il sistema: \[ a_i \frac{d^iy(t)}{dt^i} + \dots + a_1 \frac{dy(t)}{dt} + a_0y(t) = b_j\frac{d^ju(t)}{dt^j} + \dots + b_1\frac{du(t)}{dt} + b_0 u(t) \]

	allora \[ W(s) = \frac{b_js^j + \dots + b_1s + b_0}{a_is^i + \dots + a_1s + a_0} = \vec{C}(s\mathbb{I} - \vec{A})^{-1}\vec{B} + \vec{D} \]

	Il polinomio caratteristico è: \[P(s) = det(s\mathbb{I} - \mathbb{A})\] le cui soluzioni $\lambda_1, \dots, \lambda_n $ sono gli autovalori della matrice $\mathbb{A}$.


	\subsection*{Modi del sistema}
	Date $p_0, p_1, \dots, p_n$le radici di $P(s)$ e $r_0, r_1, \dots, r_n$ le relative molteplicità, definiamo i modi del sistema come \[m_{i, j} = \frac{t^j}{j!}e^{p_i}t\] e si classificano: \begin{itemize}
		\item \textbf{convergenti a 0} se $Re\lbrace p_i\rbrace < 0$;
		\item \textbf{limitati} se $Re\lbrace p_i\rbrace \leq 0$ e i modi relativi a $p_i$ sono semplici ($r_i = 1$);
		\item \textbf{divergenti} altrimenti.
	\end{itemize}
	\noindent\hrulefill
	\section*{Stabilità}
	\textbf{Stato di equilibrio} $x_e$ tale che, dato $\dot{x}(t) = f(x(t))$, vale $f(x_e) = 0$.

	\noindent
	Il sistema è \textbf{stabile} se tutti i modi sono limitati, \textbf{asintoticamente stabile} se tutti i suoi modi sono convergenti a 0.

	\subsection*{Stabilità alla Lyapunov}
	Sia l'origine un punto di equilibrio ($V: W \to \mathbb{R}$ definita positiva).
	\begin{itemize}
		\item $\dot{V}: W \to \mathbb{R}$ semi-definita negativa $\rightarrow$ origine stabile;
		\item $\dot{V}: W \to \mathbb{R}$ definita negativa $\rightarrow$ origine asint. stabile;
	\end{itemize}

	\subsection*{Criterio ridotto di Lyapunov}
	Data $J$ matrice Jacobiana del sistema, diciamo che \begin{itemize}
		\item l'origine è un punto di equilibrio stabile se $\forall\lambda_i$ di $\mathbb{A}$, $Re\lbrace\lambda_i\rbrace < 0$;
		\item l'origine NON è un punto di equilibrio stabile se $\exists\lambda_i$ di $\mathbb{A}$, $Re\lbrace\lambda_i\rbrace \geq 0$;
	\end{itemize}



\end{document}