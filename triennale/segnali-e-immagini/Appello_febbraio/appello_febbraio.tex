\documentclass[a4paper,oneside]{scrbook}
\usepackage[italian]{babel}
\usepackage[utf8]{inputenc}
\usepackage{amsmath}
\usepackage{amssymb}
\usepackage{mathtools}
\usepackage{tikz}
\usetikzlibrary{arrows,calc,patterns,angles,quotes}
\usepackage{pgfplots}


\pagenumbering{gobble}
\setcounter{chapter}{1}

\begin{document}
\section*{Domanda 1}
\paragraph{Enunciare e descrivere le proprietà dei sistemi lineari tempo-invarianti (LTIS).}\
\newline
I sistemi LTIS (\textit{Linear Time Invariant Systems}) sono tutti i sistemi lineari, tempo-invarianti. Questi hanno le seguenti proprietà:
\begin{itemize}
	\item \textbf{Linearità}\\
	Dato un sistema con input $f(t)$ e output $y(t)$, il sistema è lineare se gode delle seguenti caratteristiche:
	\begin{itemize}
		\item \textbf{Additività}\\
		Se $f_1 \rightarrow y_1$ e $f_2 \rightarrow y_2$ allora $f_1+f_2 \rightarrow y_1+y_2$
		\item \textbf{Omogeneità}\\
		Se $f_1 \rightarrow y_1$ allora $a_1\cdot f_1 \rightarrow a_1 \cdot y_1$
		\item \textbf{Sovrapposizione}\\
		Se $a_1 \cdot f_1 + a_2 \cdot f_2$ allora $a_1 \cdot y_1 \rightarrow + a_2 \cdot y_2$
	\end{itemize}
	\item \textbf{Tempo-invarianza}\\
	Ogni sistema può essere espresso sotto forma di equazione differenziale del tipo:
	\begin{equation*}
		a_n\frac{d^nv(t)}{dt^n} + \ldots + a_1\frac{dv(t)}{dt} + a_0v(t) = b_m\frac{d^mu(t)}{dt^m} + \ldots + b_1\frac{du(t)}{dt} + b_0u(t)
	\end{equation*}
	e i coefficienti $a_n, \ldots, a_1, a_0$ ne determinano la tempo-invarianza.
	Nel caso di sistemi lineari, i coefficienti diventano delle costanti ed il sistema risponde \textit{linearmente} al segnale in input.
\end{itemize}

\paragraph{Illustrare la risposta ‘‘zero-input’’ e ‘‘zero-state’’ di un LTIS indicandone le condizioni di applicabilità e la relazione con la risposta totale.}\
\newline
I sistemi LTIS grazie alla linearità hanno una risposta esprimibile come la somma di due componenti separate: la risposta \textbf{zero-input} e la risposta \textbf{zero-state}.
La prima indica l'output del sistema quando in input non è presente alcun segnale.
La seconda invece è la risposta del sistema ignorando le condizioni iniziali (calcolate dalla risposta zero-input).
Sempre grazie alla linearità del sistema, possiamo esprimere la \textbf{risposta totale} come la somma della zero-input con la zero-state:
\begin{equation*}
	\text{RISPOSTA TOTALE} = \text{RISPOSTA ZERO-INPUT} + \text{RISPOSTA ZERO-STATE}
\end{equation*}

\paragraph{Illustrare il procedimento per il calcolo delle radici caratteristiche di un sistema LTIS.}\
\newline
Possiamo raggruppare l'equazione differenziale che esprime un sistema LTIS in questo modo:
\begin{align*}
a_n\frac{d^nv(t)}{dt^n} + \ldots + a_1\frac{dv(t)}{dt} + a_0v(t) &= b_m\frac{d^mu(t)}{dt^m} + \ldots + b_1\frac{du(t)}{dt} + b_0u(t)\\
(D^n+a_{n-1}D^{n-1}+\ldots+a_1D^1+a_0)v(t)&=(b_mD^m+b_{m-1}D^{m-1}+\ldots+b_1D^1+b_0)u(t)\\
Q(D)v(t)&=P(D)u(t)
\end{align*}
Le \textit{radici caratteristiche} del sistema sono i coefficienti ricavati dalla soluzione della risposta zero-input del sistema, e non essendo dipendenti dall'input
definiscono il sistema stesso. La risposta zero-input è data dalla soluzione dell'equazione:
\begin{equation*}
	Q(D)v(t)=0.
\end{equation*}
Affinché l'equazione risulti zero, tutte le soluzioni devono essere della stessa forma e ciò è possibile solo con soluzioni del tipo $e^{\lambda t}$.
\begin{align*}
	(\lambda^n+a_{n-1}\lambda^{n-1}+\ldots+a_1\lambda+a_0)e^{\lambda t}&=0\\
	\lambda^n+a_{n-1}\lambda^{n-1}+\ldots+a_1\lambda+a_0&=0
\end{align*}
Tutte le $\lambda^n,\ldots,\lambda$ sono le radici caratteristiche del sistema.

\paragraph{Descrivere gli stati di stabilità di un sistema LTIS e descrivere le relative condizioni in relazione alle radici caratteristiche giustificando la risposta.}\
\newline
In generale la stabilità di un sistema è data dal fatto che dopo una perturbazione di lunghezza e ampiezza finita esso ritorni e converga verso un valore al tendere
di $t \rightarrow \infty$ invece di divergere.

Supponiamo quindi di avere un sistema LTIS, con una certa risposta zero-input. A questo sistema diamo in input un segnale $f(t)$ per un certo tempo.
Se dopo un po' il sistema ritorna nello stato iniziale, dando la stessa risposta zero-input precedente, allora il sistema è stabile.
Più precisamente, se le radici caratteristiche tendono a zero per $t \rightarrow \infty$, allora possiamo dire
che il sistema è \textbf{asintoticamente stabile}. Qualora invece non tendano a zero, ma tendano ad una costante, il sistema si definisce \textbf{marginalmente stabile}.
\begin{align*}
	y(t)=&\sum_{j=1}^{u}e^{\lambda jt}c_j\\
	\lim_{t\rightarrow +\infty} e^{\lambda jt}=&\begin{cases}
													0 \text{ se } RE{\lambda_j<0}\\
													+\infty \text{ se } RE{\lambda_j>0}
												\end{cases}
\end{align*}

\section*{Domanda 2}
\paragraph{Dati due segnali $x(t)$ e $f(t)$ a tempo continuo, fornire la definizione del coefficiente di correlazione illustrandone il significato e specificandone gli estremi di variabilità.}\
\newline
\paragraph{Considerati i segnali:
	$$x_1(t)=rect(T), x_2(t)=A*rect(T), x_3(t)=\sin(t)$$
	Tra quali coppie di segnali il coefficiente di correlazione sarà più elevato? Giustificare la risposta}\
\newline
Dati due vettori $x$ e $y$ si definiscono \textit{simili} qualora $x$ avesse una ‘‘larga componente’’ su $y$. Si può quantificare la somiglianza tra i vettori con
un \textbf{coefficiente di correlazione}, definito come:
\begin{equation*}
	c=\frac{x\cdot y}{|x|\cdot|y|}=\cos\theta
\end{equation*}
Più è stretto l'angolo dato da $\theta$ ($c=1$) più la somiglianza è grande.
Lo stesso ragionamento è possibile farlo per due segnali $x(t)$ e $f(t)$ dato che possono essere espressi in termini di vettori. Il coefficiente di correlazione tra
$x(t)$ e $f(t)$ è dato quindi da:
\begin{equation*}
	c_n=\frac{1}{\sqrt{E_fE_x}}\int_{-\infty}^{+\infty}f(t)x(t)dt
\end{equation*}
Se $c_n=1$ allora i segnali sono molto simili, se non identici. Se $c_n=0$ i segnali sono di forma differente, mentre se $c_n=-1$ i segnali sono opposti.
La correlazione così definita non funziona su segnali uguali disgiunti nel tempo, dato che risulterebbe $0$. Per definire correttamente la correlazione tra due segnali
disgiunti è necessario considerare il possibile ritardo, per questo definiamo la \textbf{correlazione incrociata}:
\begin{equation*}
	\Psi_{f_y}(t)=\int_{-\infty}^{+\infty}f(\tau)g(\tau-t)d\tau
\end{equation*}

Prendendo in considerazione i segnali $$x_1(t)=\text{word or phrase}rect(T), x_2(t)=A*rect(T), x_3(t)=\sin(t)$$ la correlazione tra $x_1$ e $x_2$ avrà un valore alto
visto che sono lo stesso segnale, solamente che $x_2$ presenta una costante $A$.
Entrambi $x_1$ e $x_2$ rispetto a $x_3$ hanno correlazione $0$, dato che sono completamente differenti.

\paragraph{Data la base completa $\{x_n(t), n=1, \ldots, \infty\}$ e dato il segnale $f(t)$, fornire la rappresentazione di $f(t)$ nella base $\{x_n(t)\}$ illustrando il procedimento.}\
\newline
Data una base completa $\{x_n(t)\}$ il segnale $f(t)$ può essere espresso come una combinazione lineare:
\begin{equation*}
	f(t) \simeq c_1x_1(t)+c_2x_2(t)+\ldots+c_nx_n(t)=\sum_{i=1}^{n}c_ix_i(t)
\end{equation*}
dove l'errore è dato da
\begin{equation*}
	e(t)=f(t)-\sum_{i=1}^{n}c_ix_i(t)
\end{equation*}

\paragraph{Dato il segnale periodico $f(t)$, esprimere i coefficienti della corrispondente serie di Fourier e descrivere il significato.}\
\newline
Dato un segnale $f(t)$, possiamo rappresentarlo come somma di funzioni seni e coseni in questo modo:
\begin{equation*}
	f(t) \cong a_0 + \left( \sum_{n=0}^{+\infty} a_n \cos(n\omega_0t+\theta_n)+b_n\sin(n\omega_0t+\theta_n) \right)
\end{equation*}
dove $\omega_0=\frac{2\pi}{T}$, con $T$ periodo di $f(t)$. I valori assunti da $\theta_n$ nella serie definiscono quello che viene detto \textbf{spettro di fase},
che informa sulla quantità di ogni frequenza che compone il segnale.
I coefficienti $a_n$ e $b_n$ (insieme a $n\omega_0t$) compongono lo \textbf{spettro in ampiezza} del segnale.

\paragraph{In quale caso la serie di Fourier di un segnale $f(t)$ è reale? Giustificare la risposta.}\
\newline
La parte reale della trasformata di Fourier di un segnale indica le sue componenti pari, quella immaginaria invece esprime le componenti dispari del segnale stesso.
Inoltre segnali pari esprimibili come somme di coseni producono una trasformata solamente reale mentre al contrario segnali dispari possono essere espressi da sinusoidi
e hanno solamente parte immaginaria.

\section*{Domanda 3}
\paragraph{Considerate le immagini in Figura 1(A) e 1(B). Indicare quale tipo di operazione sia stata effettuata per trasformare la prima nella seconda e descrivere tale
operazione nel dettaglio.}\
\newline
In una figura è stata applicata un operazione chiamata \textit{histogram equalization}. Questo tipo di operazione prevede di calcolare per l'immagine il corrispondente
istogramma che sull'asse delle ascisse inserisce tutte le possibili gradazioni di grigio mentre sull'ordinata si indicano il numero di pixel con quella gradazione.
Il \textit{re-arrangement} dell'istogramma prevede il calcolo di una distribuzione dei valori dei pixel secondo un modello statistico. Una volta applicata applicata la
distribuzione, l'istogramma risulta \textit{stretched} se i valori sono stati limitati ad gruppo ristretto di gradazioni, \textit{shrinked} nel caso contrario.
Nel primo caso quindi l'immagine risultante avrà meno gradazioni di grigio e alcune più ‘‘ardenti’’, mentre nel secondo caso la distribuzione è più omogenea per cui
l'immagine risulterà più uniforme nelle varie gradazioni.

\paragraph{Porre in corrispondenza gli istogrammi (C) e (D) con le immagini (A) e (B) e giustificare la risposta.}\
\newline
Associazioni: $C \rightarrow A$, $D \rightarrow B$.\\
Vedi risposta precedente per giustificazione.

\section*{Domanda 4}
\paragraph{Fornire la definizione di a) risposta impulsiva e b) funzione di trasferimento di un filtro e specificare quale sia la relazione tra le due.}\
\newline
La risposta impulsiva di un sistema è l'output del sistema quando in input viene passata la funzione \textbf{delta di Dirac} (detta anche impulso unitario):
\begin{equation*}
	\int_{-\infty}^{+\infty} \delta(t)dt=1
\end{equation*}
Ogni segnale in input può essere rappresentato come una somma di impulsi unitari:
\begin{equation*}
	f(t)=\int_{-\infty}^{+\infty} \delta(t-\tau)f(\tau)d\tau
\end{equation*}
Visto che $\delta(t)$ in input genera la risposta impulsiva $h(t)$ allora anche l'output di un generico segnale $f(t)$ può essere espresso come un segnale
composto dal prodotto della risposta impulsiva con l'input stesso:
\begin{equation*}
	y(t)=\int_{-\infty}^{+\infty}f(\tau)h(t-\tau)d\tau
\end{equation*}
Questo integrale è detto integrale di \textit{convoluzione}.

la funzione di trasferimento di un sistema è una funzione che mette in relazione l'input e l'output nel dominio delle frequenze. Essa non è altro che la
\textit{trasformata di Fourier} della \textit{risposta impulsiva} di un sistema LTI.
\begin{equation*}
	H(s)=\frac{P(S)}{Q(S)}
\end{equation*}

\paragraph{Descrivere ed illustrare graficamente la risposta impulsiva e la funzione di trasferimento di un filtro passa basso ideale in due dimensioni.}\
\newline
Un filtro passa-basso in due dimensioni può essere espresso come un'eliminazione circolare applicata alla trasformata dell'immagine. Viene troncato un cerchio
di raggio $r$ sull'immagine della trasformata e non vengono considerate le frequenze all'esterno del cerchio (che sono appunto le frequenze basse). A quel punto
si ricostruisce l'immagine con l'inversa della trasformata e si ha l'immagine filtrata con un filtro-basso di dimensione $r$.
La funzione di trasferimento è data da:
\begin{equation*}
	H(u,v)=
	\begin{cases}
		1 \quad \text{ se } D(u,v)\leq D_0\\
		0 \quad \text{ se } D(u,v)>D_0
	\end{cases}
\end{equation*}

\end{document}