\documentclass[a4paper, 10pt]{article}
%librerie / userpackage
\usepackage[italian]{babel}
\usepackage[T1]{fontenc}
\usepackage[utf8]{inputenc}
\usepackage{amsmath}
\usepackage{amsfonts}
\usepackage{amssymb}
\usepackage{hyperref}
\usepackage{graphicx}
\usepackage{palatino}
\usepackage[margin=3cm]{geometry}

\hypersetup{
	hidelinks, 
	colorlinks = true,
	linkcolor = black,
}

\begin{document}
		\clearpage
	\begin{titlepage}
		\centering
		\vspace*{\fill}
		{\scshape\LARGE Università degli Studi di Verona \par}
		\vspace{1.5cm}
		\line(1,0){150} \\
		{\huge\bfseries Fisica 2\par}
		\line(1,0){150} \\
		\vspace{0.5cm}
		{\scshape\Large Riassunto dei principali argomenti\par}
		\vspace{2cm}
		{\Large\itshape Matteo Marzio\par}
		\vspace{1cm}
		\vspace{5cm}
		\vspace*{\fill}
		% Bottom of the page
		{\large \today\par}
	\end{titlepage}
	\pagenumbering{arabic}
	\thispagestyle{empty}
	\newpage
	\tableofcontents
	
	\pagenumbering{roman}
	\newpage
	\section{Elettrostatica nel Vuoto}
		\subsection{Carica Elettrica}
			La carica elettrica e' una grandezza fisica scalare dotata di segno, ed e' una proprietà fondamentale della
			materia. \\
			La carica elettrica e' un tipo di carica ed e' responsabile dell'interazione elettromagnetica, essendo
			sorgente del campo elettromagnetico. \\
			E' una grandezza quantizzata, ossia esiste solo in forma di multipli di una quantità fondamentale: la carica
			dell'elettrone.\\ 
			Essa viene misurata in \textbf{Coulomb} e vale 
				\[ -e = -1,602176634 \cdot 10^{-19} C = - \frac{1C}{6,241509074 \cdot 10^{18}} \]
			La carica di un protone, uno dei costituenti fondamentali del nucleo insieme al neutrone, viene considerata
			positiva ed e' indicata con $+e$.
		\subsection{Struttura della Materia}
		\paragraph*{Atomo} Gli atomi sono le particelle che compongono la materia:essi sono costituiti a loro volta da
						   tre tipi di particelle più piccole: i protoni, i neutroni e gli elettroni. 
						   L'atomo consiste in un nucleo formato da protoni e neutroni, circondato da un involucro di
						   elettroni.
		\paragraph*{Carica delle Particelle} I protoni e gli elettroni possiedono una proprietà elettrica chiamata
										    \textbf{carica}. Essa e' una grandezza fisica che può avere segno positivo o
										    negativo: le cariche dello stesso segno si respingono, mentre le cariche 
										    di segno diverso si attraggono.
		\subsection{Forza di Coulomb}
		\paragraph*{Descrizione} La forza di Coulomb, descritta dalla Legge di Coulomb, e' la forza esercitata da un 
								campo elettrico su una carica elettrica. Si tratta della forza che agisce tra oggetti
								elettricamente carichi, ed e' operativamente definita dal valore dell'interazione tra 
								due cariche elettriche puntiformi e ferme nel vuoto.
		\paragraph{Definizione} Si considerino due cariche puntiformi interagenti, il cui valore e' indicato con $q_1$ e 
							   $q_2$, nelle posizioni $r_1$ e $r_2$. La forza di Coulomb e' la forza esercitata da $q_2$ 
							   su $q_1$ e ha l'espressione
							   \[ F = k \cdot q_1 \cdot q_2 \cdot \frac{r_1 - r_2 }{\vert \vert r_1 - r_2 \vert \vert ^3}
							   \hspace{0.7cm} [N] \]
							   dove $k$ e' la \textbf{costante di Coulomb}, che e' pari a 
							   \[k = \frac{1}{4\pi \epsilon_0} \hspace{0.7cm} [N m^2 C^{-2}]\] 
							   con $\epsilon_0$ la costante dielettrica nel vuoto, equivalente a 
							   \[ \epsilon_0 = 8,854 \hspace{0.7cm} N^{-1} m^{-2} C^2 \]
							   La forza tra due cariche e' proporzionale al prodotto dei loro valori $q_1$ e $q_2$,
							   inversamente proporzionale al quadrato della loro distanza ed e' diretta come la congiungente
							   $r_1 - r_2$.
		\subsection{Campo Elettrico}
			Il campo elettrico e' un campo di forze generato nello spazio dalla presenza di una o più cariche elettriche o 
			di un campo magnetico variabile nel tempo. \\
			Insieme al campo magnetico esso costituisce il campo elettromagnetico, 
			responsabile dell'interazione elettromagnetica. Introdotto da Michael Faraday, il campo elettrico si propaga 
			alla velocità della luce ed esercita una forza su ogni oggetto elettricamente carico. \\
			Esso si misura in newton su coulomb $(N C^{-1})$ o volt su metro $(V m^{-1})$.
		\subsection{Campo Elettrostatico}
			Se un campo elettrico e' generato dalla sola distribuzione stazionaria di carica spaziale, il campo elettrico 
			e' detto elettrostatico ed e' conservativo. \\
			Nel vuoto, il campo elettrico $\textbf{E}$ in un punto dello spazio e' definito come la forza per unita' 
			di carica elettrica positiva alla quale e' soggetta una carica puntiforme $q$, detta carica "di prova", 
			se posta nel punto:
			\[ E = \lim_{q \to 0} \frac{F}{q} \]
			Dalla legge di Coulomb segue che una carica $Q$ posta in $r^{'}$ genera un 
			campo elettrica che in un punto qualsiasi $r$ e definito dalla seguente espressione:
			\[ E(r) = \frac{Q}{4 \pi \epsilon_0} \cdot \frac{r - r^{'}}{\vert \vert r - r^{'} \vert \vert} \]
			Da questa formula si può ricavare la forza elettrica tra due cariche $q_1$ e $q_0$, che e' uguale a:
			\[ \overrightarrow{F} = q_0 \cdot \frac{q_1}{4 \pi \epsilon_0 r^2_{10}} \overrightarrow{u_{10}}
			= q_0 \overrightarrow{E}(\overrightarrow{r_0}) \]
			con $\overrightarrow{u_{10}}$ il vettore spostamento con direzione da $q_1$ a $q_0$, 
			$r_{10}$ la distanza tra la carica $q_1$ alla carica $q_0$.
		\subsection{Energia Potenziale Elettrostatica}
			L'energia potenziale elettrica, anche detta energia potenziale elettrostatica, e' l'energia potenziale del campo
	    		elettrostatico. \\
	    		Si tratta dell'energia posseduta da una distribuzione di carica elettrica, ed e' legata alla 
	    		forza esercitata dal campo generato dalla distribuzione stessa. \\
	    		L'energia elettrostatica e' descritta come il lavoro necessario per portare un sistema di cariche elettriche in
	    		una data configurazione ed e' definita come:
	    		\[ W_n = \frac{1}{2} \sum_{i,j=1}^n \frac{q_i q_j}{4 \pi \epsilon_0 r_{ij}} \]
	    		\newpage
	    	\subsection{Potenziale Elettrico}
	    		Data una regione di spazio in cui e' presente un campo elettrico conservativo, si definisce potenziale elettrico
	    		in un punto il valore dato dal rapporto dell'energia potenziale elettrica (o elettrostatica) rilevato da una
	    		carica di prova : 
	    		\[ V = \frac{U}{q_{prova}} \]
	    		Il potenziale dunque e' una quantità scalare e \textbf{non dipende} dal valore della carica di prova. \\
	    		Esso viene misurato in volt: tra due punti $A$ e $B$ di una regione di spazio sede di un campo elettrico c'è una
	    		differenza di potenziale di $1V$ se la forza elettrica compie il lavoro di un Joule per portare una carica di 
	    		un Coulomb da $A$ a $B$. \\
	    		Il lavoro $dW$ svolto dal campo elettrico $E_0$ per un percorso infinitesimo $ds$ su una carica $q$ e' dato da:
	    		\[ dW = qE_0 \cdot ds \]
	    		e per calcolare il lavoro lungo una linea $l$ da un punto $A$ ad un punto $B$:
	    		\begin{equation*}
	    			\begin{split}
	    			 	W &= W_A - W_B = \int_{A}^{B} qE_0 \cdot ds = \\
	    			 	  &= \int_{r_A}^{r_B} \frac{q_0 Q}{4 \pi \epsilon_0 r^2} \cdot u_r dl = \\
	    			 	  &= \frac{q_0 Q}{4 \pi \epsilon_0} \cdot (\frac{1}{r_A} - \frac{1}{r_B}) \\
	    			\end{split}
			\end{equation*}	    		
	    		Consideriamo ora  $W(\infty) = 0$ : 
	    		\[ W(r) = \frac{q q_0}{4 \pi \epsilon_0 r} = -\int_{\infty}^{r} \overrightarrow{E} \cdot d\overrightarrow{r} = 
	    		W(r) - W(\infty) = \int_{r}^{\infty} F \cdot ds \]
	    		Se il lavoro e' positivo, allora  esso rappresenta il lavoro per costruire il sistema $(r \rightarrow \infty)$, 
	    		altrimenti se e' negativo rappresenta il lavoro del sistema per disgregarsi $(r \leftarrow \infty)$.
	    	\subsection{Superfici Equipotenziali}
	    		Si definiscono \textbf{superfici equipotenziali} per il potenziale elettrico le superfici in ogni 
	    		punto delle quali il potenziale elettrico assume lo stesso valore, ovvero 
	    		sono le superfici perpendicolari alle linee di campo dove il lavoro e' nullo:
	    		\[ dL = - dV = \overrightarrow{E} \cdot d \overrightarrow{s} = 0 \]
	    	\subsection{Linee di Campo}
	    		Una linea di campo, anche detta linea di forza, e' una curva ideale che ha come tangente in ogni punto la
	    		direzione del vettore del campo stesso. \\Per ogni punto passa una sola linea di campo. \\
	    		Esse non si incrociano per nessun punto e non sono chiuse (altrimenti il flusso del campo $E$ sarebbe nullo)
		\subsection{Flusso del Campo}
			Si consideri la superficie infinitesima $dS$ in un campo elettrico $\overrightarrow{E}$ e la si consideri
			approssimativamente come piana. Alla superficie posso associare un vettore di modulo uno (versore) 
			$\widehat{n}$ che ne identifica l'orientamento, scegliendo il vettore perpendicolare alla superficie. \\
			Se la superficie e' abbastanza piccola, posso considerare il campo elettrico costante in modulo direzione e 
			verso su tutta la superficie. \\
			Il flusso del campo elettrico attraverso questa piccola superficie e' definito come il prodotto del modulo $E$ del 
			campo elettrico per l'estensione della superficie $dS$ per il coseno dell'angolo che la perpendicolare (o normale)
			alla superficie fa con il campo elettrico. Il flusso e' quindi un prodotto scalare.:
			\[ d \Phi (\overrightarrow{E}) = \overrightarrow{E} \cdot \widehat{n} = E \cdot dS \cdot cos(\theta)\]
			Il flusso si può anche definire in termini di linee di campo: il numero di linee del campo elettrico che attraversano
			una superficie e' proporzionale alla superficie e all'intensità del campo elettrico. \\ 
			Il flusso del campo elettrico e' quindi proporzionale alle linee di campo che attraversano una superficie.
		\subsection{Teorema di Gauss}
			Si consideri una superficie sferica al centro del quale e' racchiusa una carica $Q$ positiva che genera un campo. \\
			Ricaviamo delle porzioni (anche non uguali) di superficie tali che:
			\[ \Phi_S(\overrightarrow{E}) =  \Phi_{S_1}(\overrightarrow{E}) + \cdots +  \Phi_{S_n}(\overrightarrow{E}) \]
			Possiamo affermare che il flusso complessivo e' dato dalla somma di tutti i singoli flussi calcolati per ogni $S$.\\
			Se calcolassimo il flusso di una singola superficie, potremmo notare che il vettore $\overrightarrow{E}$, 
			partente dalla superficie, e il versore $\widehat{n}$, che essendo per definizione perpendicolare alla superficie,
			coincidono. \\ Cio' significa che l'angolo $\theta = 0$ e quindi che:
			\[ \Phi_{\Delta S_n}(\overrightarrow{E}) = E \cdot \Delta S_n \cdot cos(0) = E \cdot \Delta S_n \] 
			Fatte queste considerazioni, riscriviamo la formula del flusso totale, sostituendo ogni singolo flusso con la formula
			 appena ricavata:
			\[ \Phi_S(\overrightarrow{E}) = E \cdot \Delta S_1 + \cdots + E \cdot \Delta S_n =
			E \cdot (\Delta S_1 + \cdots + \Delta S_n) \]
			Raccogliendo il fattore $E$, e' facile notare come la somma delle superfici e' in realtà l'intera superficie della
			sfera. Riscriviamo quindi la formula scrivendo al posto della somma la formula per la superficie di una sfera, e al
			posto di $E$ la formula del campo:
			\[ \Phi_s(\overrightarrow{E}) = \frac{1}{4 \cdot \pi \cdot \epsilon_0} \cdot \frac{Q}{r^2} \times 
			4 \cdot \pi \cdot r^2 = \frac{Q}{\epsilon_0} \]
			Abbiamo quindi ricavato il \textbf{teorema di Gauss} :
			\[ \Phi_s(\overrightarrow{E}) = \oint_{S} E \cdot dS = \frac{Q}{\epsilon_0} \]
		\subsection{Teorema di Gauss in diverse superfici}
			\paragraph*{Sfera di raggio r} Scegliamo come superficie chiusa $S$ una sfera di raggio $r$ concentrica con la carica 
		 	e calcoliamo il flusso attraverso essa:
			\begin{equation*}
				\begin{split}
					\Phi_E &= \oint_S \overrightarrow{E} \cdot d\overrightarrow{S} = \oint_S \frac{1}{4 \pi \epsilon_0} 
					\cdot \frac{q}{r^2} \cdot d \overrightarrow{S} \\
					&= \frac{1}{4 \pi \epsilon_0} \cdot \frac{q}{r^2} \cdot \oint_S \widehat{r} \cdot \widehat{n} \cdot d
					\overrightarrow{S}
				\end{split}
			\end{equation*}
			Visto che $E$ rimane costante, lo posso portare fuori dalla circuitazione ($\oint$). 
			\begin{equation*}
					\Phi_E = \frac{1}{4 \pi \epsilon_0} \cdot \frac{q}{r^2} \cdot \oint_S d\overrightarrow{S} = 
					\frac{1}{4 \pi \epsilon_0} \cdot \frac{q}{r^2} \cdot (4 \pi r^2) = \frac{q}{\epsilon_0}
			\end{equation*}
			Quindi, indipendentemente dal raggio della sfera,
			\[ \Phi_E = \frac{q}{\epsilon_0} \]
			\paragraph*{Superficie generica}
			Se la superficie $S'$ e' generica, sarà sempre possibile trovare una sfera $S$ completamente contenuta in $S'$. 
			Non essendoci altre cariche, il numero delle linee di campo attraverso $S'$ sarà necessariamente uguale al numero delle
			linee di campo attraverso $S$. Quindi: \[ \Phi_E(S') = \Phi_E(S) = \frac{q_{int}}{\epsilon_0} \]
			\paragraph*{Carica esterna alla superficie} Data una superficie esterna alla carica $q$, consideriamo le linee di campo 
			in un piccolo cono con vertice $q$. L'intersezione di questo cono con la superficie $S$ determina due piccole aree 
			$\Delta S_1$ e $\Delta S_2$ attraversate dallo stesso numero di linee di campo:
			\begin{equation*}
			\begin{split}
				\Delta \Phi_{cono} &= \Delta \Phi_1 + \Delta \Phi_2 \\
				&= \vert \Delta \Phi_1 \vert - \vert \Delta \Phi_2 \vert = 0
			\end{split}
			\end{equation*}
			Questo e' vero per ogni piccolo cono con vertice in $q$ che interseca la superficie $S$
			\[ \Phi_E = \sum \Delta \Phi_{cono, i} = 0 \]
			Per il principio di sovrapposizione, il risultato precedente e' generalizzabile a qualsiasi distribuzione di cariche.
		\subsection{Teorema di Coulomb}
			\paragraph*{Descrizione}
			Il teorema di Coulomb e' una relazione che permette di determinare l'intensità del campo elettrico in prossimità della
			superfici di un corpo conduttore conoscendo la densità di carica in quel punto. Permette quindi di calcolare il campo
			elettrico in prossimità della superficie una volta che si conosce la densità con cui vi sono distribuite le cariche.
			\paragraph*{Definizione}Dato un corpo conduttore la cui superficie sia caratterizzata da una densità' superficiale di 
			carica $\sigma$, il campo elettrico prodotto in prossimità della superficie e':
			\[ E = -\frac{\delta V}{\delta n} = \frac{\sigma}{\epsilon_0} \widehat{n} \]
		\subsection{Discontinuità del campo elettrico}
			Il campo elettrico e' definito come il rapporto tra la forza che il campo esercita su una carica e la carica stessa,
			ovvero: \[ \overrightarrow{E} = \frac{\overrightarrow{F}}{q} \]
			In condizioni statiche, il campo elettrico e' conservativo, quindi il potenziale e' continuo. 
			Si consideri una sfera conduttiva carica. Dato che la sfera e' conduttiva, se poniamo una carica positiva, le cariche al 
			suo interno tenderanno ovviamente a respingersi e conseguentemente, potendosi muovere, tendono ad allontanarsi il più
			possibile tra loro. Dunque la sfera carica genera un campo elettrico per effetto delle cariche superficiali, 
			ma all'interno il campo e' nullo. Esiste quindi una discontinuità nel campo. \\
			Per calcolare se la componente \textbf{normale} del campo $\overrightarrow{E}$ ha discontinuità nell'attraversare 
			una superficie carica, bisogna usare il teorema di Coulomb, ovvero:
			\[ E_{\bot} = \frac{\sigma}{\epsilon_0} \]
			Si ricorda inoltre che $E_{\parallel} = 0$
		\subsection{Applicazioni di Gauss}
			\subsubsection{Distribuzione piana infinita di carica con densità $\sigma$}
				Se dividiamo lo spazio omogeneo ed isotropo (ovvero stesse proprieta' in tutte le direzioni) con un piano
				verticale, la condizione che zone a destra e sinistra del piano devono essere per simmetria indistinguibili 
				impone che:
				\begin{itemize}
					\item Le linee di campo devono essere perpendicolari al piano carico
					\item Nei punti ad uguale distanza, a destra e sinistra del piano, il campo deve avere lo stesso valore
				\end{itemize}
				Scegliamo quindi come superficie chiusa per calcolare il flusso un cilindro retto di area di base $A$, con asse
				perpendicolare al piano e basi equidistanti dal piano, denominate $A_1$ e $A_2$.
				Segue quindi che :
				\begin{equation*} \begin{split}
					\underset{cilindro}{\oint} \overrightarrow{E} \cdot d\overrightarrow{S} &= 
					\underset{A_1}{\int} \overrightarrow{E} \cdot d\overrightarrow{S} + \underset{sup.lat}{\int} 
					\overrightarrow{E} \cdot 
					d\overrightarrow{S} + \underset{A_2}{\int} \overrightarrow{E} \cdot d\overrightarrow{S} = 
					\frac{q_{int}}{\epsilon_0} \\
					\overrightarrow{E}_1 \cdot d\overrightarrow{S} &= \overrightarrow{E}_1 cos(0)dS \Rightarrow 
					\underset{A_1}{\int}\overrightarrow{E}_1 \cdot d\overrightarrow{S} = E_1 \underset{A_1}{\int}
					 d\overrightarrow{S} = E_1 A \\
					\overrightarrow{E}_2 \cdot d\overrightarrow{S} &= \overrightarrow{E}_2 cos(0)dS \Rightarrow 
					\underset{A_2}{\int}\overrightarrow{E}_2 \cdot d\overrightarrow{S} = E_2 \underset{A_2}{\int} 
					d\overrightarrow{S} = E_2 A 
				\end{split} \end{equation*} 
				Sulla superficie laterale 
				$\overrightarrow{E} \bot \overrightarrow{n} \Rightarrow \underset{sup.laterale}{\int \overrightarrow{E} \cdot 
				d\overrightarrow{S}} = 0$
				\[ \underset{cilindro}{\oint} \overrightarrow{E} \cdot d\overrightarrow{S} 
				= E_1 A + E_2 A = \frac{q_{int}}{\epsilon_0} = \frac{\sigma A}{\epsilon_0}\]
				Posto $E_1 = E_2 = E$,
				\[ 2EA = \frac{\sigma A}{\epsilon_0} \Rightarrow E = \frac{\sigma}{2\epsilon_0}\]
				Conclusione : \textbf{il campo e' uniforme e vale} \[\overrightarrow{E} = 
				\frac{\sigma}{2\epsilon_0}	\overrightarrow{n}\]
			\subsubsection{Doppia distribuzione piana infinita di carica di densità $\pm \sigma$}
				Questo risultato ci permette di calcolare il campo elettrico, usando il principio di sovrapposizione, generato
				da una doppia distribuzione piana infinita di carica con densità $\pm \sigma$.
				\begin{itemize}
					\item Alle estremità il campo e' nullo, visto che: 
					\[\overrightarrow{E_T} = \overrightarrow{E_+} + \overrightarrow{E_-} = 0 \]
					\item Invece all'interno della zona compresa fra i piani abbiamo un risultato diverso:
					\[\overrightarrow{E_T} = \overrightarrow{E_+} + \overrightarrow{E_-} \neq 0 \]
				\end{itemize}
				Visto che il campo e' uniforme nella zona compresa fra i piani, calcoliamo quanto vale:
				\[ E_T = 2 \frac{\sigma}{2\epsilon_0} = \frac{\sigma}{\epsilon_0} \]
				Conclusione: \textbf{il campo e' uniforme nella zone compresa fra i piani e vale} $\overrightarrow{E} =
				\frac{\sigma}{\epsilon_0} \overrightarrow{n}$ \textbf{mentre e' nullo all'esterno}.
			\subsubsection{Distribuzione sferica di carica}
				Consideriamo una sfera di raggio $R$ caricata con una carica $q$. 
				Essendo in una situazione di simmetria sferica in uno spazio omogeneo ed isotropo, questo impone che :
				\begin{itemize}
					\item Le linee di campo devono avere direzione radiale
					\item Nei punti ad uguale distanza dal centro della sfera il campo deve avere lo stesso valore
				\end{itemize}
				Con questo in mente, scegliamo come superficie chiusa per calcolare il flusso una sfera di raggio $r$ concentrica
				con la sfera carica. Segue per $r \> R$.
				\[ \Phi_E = \oint_{sfera} \overrightarrow{E} \cdot d\overrightarrow{S} = \frac{q_{int}}{\epsilon_0} \]
				dove $q_{int} = q$, e per ogni elemento $d\overrightarrow{S}$ della sfera abbiamo:
				\[ \overrightarrow{E} \cdot d \overrightarrow{S} = \overrightarrow{E} \cdot \overrightarrow{n}dS 
				= E cos(0) dS = EdS \Rightarrow \]
				\[ \Rightarrow \underset{sfera}{\int} \overrightarrow{E} \cdot d\overrightarrow{S} =
				\underset{sfera}{\int} EdS = E \underset{sfera}{\int} dS = E \cdot ( \pi r^2) = \frac{q}{\epsilon_0} \]
				Conclusione: \textbf{Il campo nei punti a distanza dal centro} $r \geq R$, e' \\
				$\overrightarrow{E}(\overrightarrow{r}) = \frac{1}{4\pi \epsilon_0}\frac{q}{r^2}\widehat{r}$, 
				ovvero uguale a quello che si otterrebbe concentrando tutta la carica nel centro della sfera. 
				In particolare, se la sfera e' conduttrice, per $r < R$ il campo e' nullo.
			\subsubsection{Doppia distribuzione sferica superficiale di carica}
				Consideriamo due sfere concentriche di raggio $R_1$ e $R_2$ e con carica $+q$ e $-q$ sulle rispettive superfici.
				Essendo ancora in una situazione di simmetria sferica, valgono ancora le considerazioni precedenti. \\
				Scegliamo come superfici chiuse per calcolare il flusso delle due sfere di raggio $r_1$ e $r_2$ concentriche 
				con la sfera.
				\begin{itemize}
					\item Per la sfera di raggio $R_1 < r_1 < R_2$ il risultato e' il medesimo al precedente, ovvero 
					\[ \overrightarrow{E}(\overrightarrow{r}) = \frac{1}{4\pi \epsilon_0}\frac{q}{r^2}\widehat{r} \]
					\item Per la sfera di raggio $r_2 > R_2$, procedendo come prima giungiamo a 
					\[ \Phi_E = \underset{sfera2}{\int} \overrightarrow{E} \cdot d\overrightarrow{S} = \frac{q_{int}}{\epsilon_0}
					\Rightarrow E \cdot (4 \pi r^2) = \frac{q_{int}}{\epsilon_0} \]
					dove \[ q_{int} = +q - q = 0 \Rightarrow E = 0 \]
				\end{itemize}
				Conclusione: \textbf{il campo nei punti a distanza dal centro $R_1 \leq r \leq R_2$} e' \\ 
				$\overrightarrow{E}(\overrightarrow{r}) = \frac{1}{4\pi \epsilon_0}\frac{q}{r^2}\widehat{r}$, \textbf{mentre per 
				$r > R_2$ e' uguale a zero.} In particolare, se la sfera interna e' conduttrice, per $r < R_1$ e $r > R_2$ il 
				campo risulta ancora nullo.
		\subsection{Gradiente di una funzione}
		Si consideri una funzione $f(x,y,z)$ definita su un insieme aperto $A \subseteq \mathbb{R}^3$ e sia $(x_0,y_0,z_0) 
		\in A$. \\ Se esistono in $(x_0,y_0,z_0)$ sia la derivata parziale rispetto ad \[x,y,z \rightarrow 
		f_x(x_0,y_0,z_0), f_y(x_0,y_0,z_0), f_z(x_0,y_0,z_0) \] allora e' possibile costruire un 
		vettore che ha per componenti le derivate parziali: 
		\[ \nabla f(x_0,y_0,z_0) = (f_x(x_0,y_0, z_0) , f_y(x_0,y_0,z_0), f_z(x_0,y_0,z_0)) \]
		Il vettore prende il nome di gradiente della funzione $f$ valutato in $(x_0,y_0,z_0)$ o 
		ancora $\nabla f(x_0,y_0,z_0)$
		\subsection{Gradiente del Potenziale Elettrico}
			\[
				V_2 - V_1 = \int_{1}^2 \overrightarrow{\nabla} V \cdot dr = 
				- \int_{1}^2 \overrightarrow{E} \cdot d\overrightarrow{r} \Rightarrow \overrightarrow{E} = 
				- \overrightarrow{\triangledown} V 
			\]
			Dal potenziale $V(\overrightarrow{r})$ nello spazio si calcola il campo $\overrightarrow{E}(x,y,z)$.
			\[ \overrightarrow{\nabla} A = (\delta_x A, \delta_y A, \delta_z A) \]
			Essendo il campo $E$ conservativo, possiamo dire che 
			\[ \begin{cases} 
				\overrightarrow{E} = -\overrightarrow{\triangledown} V \\
				\oint E = 0 \\
				\overrightarrow{\triangledown} \times -\overrightarrow{E} = 0
				\end{cases}
			\]
			\textbf{Nota}: Queste equazioni valgono solo in \textbf{elettrostatica}!
			
		\subsection{Equazione di Poisson}
			\paragraph*{Definizione Canonica}
			Sia $\phi = \phi(x)$ una funzione definita sulla chiusura dell'insieme $U$ di $\mathbb{R}^n$ a valori 
			di $\mathbb{R}$.\\
			L'equazione di Poisson per $\phi$ ha la forma 
			\[ \vartriangle \phi = f \] dove $\vartriangle$ e' l'operatore di Laplace o laplaciano e $f$ e' definita in $U$
			a valori di $\mathbb{R}$. Nello spazio euclideo l'operatore di Laplace e' spesso denotato con $\triangledown ^2$.\\
			In coordinate cartesiane in tre dimensioni l'equazione prende la forma 
			\[  \big (  \frac{\partial^2}{\partial x^2} + \frac{\partial^2}{\partial y^2} + \frac{\partial^2}{\partial z^2} \bigr )
			\phi(x,y,z) = f(x,y,z) \]
			\paragraph*{Definizione nel Campo Elettrostatico}
			Le proprietà dell'elettrostatica si possono riassumere in una unica equazione, ovvero l'equazione di Poisson.\\
			Sia definito $\vartriangle \cdot \vartriangle = \vartriangle ^2 = \partial_x ^2 + \partial_y ^2 + \partial_z ^ 2$ come la 
			divergenza del gradiente e sia esso un operatore scalare. Allora possiamo dire che 
			\[ \vartriangle^2 V = \frac{-\sigma}{\epsilon_0}  \Leftrightarrow 
			 \frac{\partial^2}{\partial x^2} + \frac{\partial^2}{\partial y^2} + \frac{\partial^2}{\partial z^2} = 
			 \frac{-\sigma}{\epsilon_0}
			\]
			e' una equazione differenziale del secondo ordine non omogenea, la cui soluzione e' il potenziale 
			(ovvero il campo) in \textbf{ogni} punto dello spazio.
		\subsection{Equazione di Laplace}
		L'equazione di Laplace e' l'equazione omogenea associata all'equazione di Poisson. 
		L'equazione \textbf{impone} che l'operatore di Laplace di una funzione incognita sia nullo. Quindi, 
		l'equazione di Laplace non e' altro che l'equazione di Poisson messa uguale a zero:
		\[ \nabla ^2 V = 0\]
		
	\newpage
	\section{Elettrostatica nei Conduttori}
		\subsection{Conduttori in Equilibrio}
			Il conduttore può essere visto come un reticolo atomico immerso in un gas di elettroni liberi di muoversi
            all'interno del materiale. In assenza di un moto netto degli elettroni in una certa direzione, il conduttore 
			e' detto in equilibrio elettrostatico. In tale circostanza, valgono le seguenti proprietà:
			\begin{itemize}
				\item Il campo elettrico all'interno del conduttore e' ovunque nullo
				\item Un qualunque eccesso di carica su conduttore deve localizzarsi superficialmente
				\item All'esterno del conduttore, in prossimità della superficie, il campo elettrico e' perpendicolare alla
					  superficie ed ha intensità pari a $\frac{\sigma}{\epsilon_0}$, dove $\sigma$ e' la densità 
				 	  superficiale di carica.
				\item Su un conduttore di forma irregolare la carica tende ad accumularsi laddove la 
					  curvatura della superficie e' maggiore, ovvero sulle punte
			\end{itemize}
			In un conduttore sottoposto ad un campo elettrico esterno $E$ le cariche tendono a predisporsi 
			lungo le superfici in base alla direzione del campo elettrico esterno $E$.
			Quindi, si può affermare che 
			\begin{enumerate}
				\item $\overrightarrow{E_{int}} = 0$
				\item $ V = cost$ 
				\item $Q_{int} = 0$
				\item $\overrightarrow{E} = \frac{\sigma}{\epsilon_0} \cdot \overrightarrow{n}$ sulla superficie
			\end{enumerate}	
		\subsection{Induzione Elettrostatica}
			\paragraph*{Definizione} L'induzione elettrostatica e' un fenomeno tale per cui la carica elettrica all'interno di un
			 oggetto viene ri-distribuita a causa della presenza di un altro oggetto carico nelle vicinanze. 
			 A causa dell'induzione un oggetto elettricamente carico posto nelle vicinaze di un 
			 corpo conduttore neutro provoca il manifestarsi di carica su quest'ultimo, senza che avvenga contatto tra i due.
			 \paragraph*{Descrizione} Il fenomeno dell'induzione elettrostatica e' dovuto ad una ben precisa proprietà della 
			 carica: il bipolarismo. L'induzione elettrostatica si basa quindi sul principio di redistribuzione delle cariche. 
			 In equilibrio elettrostatico il campo elettrico ad un conduttore deve risultare nullo (per il teorema di Gauss). 
		\subsection{Pressione di carica superficiale}
			\paragraph*{Pressione elettrostatica} La pressione elettrostatica e' l'effetto della forza elettrostatica di Coulomb
			,quando esercitata, all'interno di un conduttore su una superficie:
			\[ p = \frac{F}{S} = \frac{\sigma^2}{2\epsilon_0} \]
			dove $\sigma$ e' la densità superficiale di carica elettrica e $\epsilon_0$ la costante dielettrica del vuoto.
			\paragraph*{Pressione elettrostatica nei condensatori}
			Poiché tra le armature di un condensatore e' presente il campo elettrostatico 
			$E = \frac{\vert \sigma \vert}{\epsilon_0}$, si può riscrivere la pressione elettrostatica come :
			\[ p = \frac{F}{S}	 = \frac{1}{2} \cdot \epsilon_0 \cdot \vert E^2 \vert \]
		\subsection{Cavità di un conduttore} 
			In un conduttore cavo le cariche si distribuiscono sulla superficie esterna, mentre il potenziale all'interno della
			cavità e' uguale a quello del conduttore, altrimenti si genererebbe un campo elettrico diverso da zero. La carica
			totale sulla superficie che delimita la cavità e nulla:
			\[ \oint_{\sum} \overrightarrow{E} \cdot n dS = \frac{Q}{\epsilon_0} = 0 \]
			Inoltre, non possono esistere cariche spazialmente separate, in quanto il campo e' conservativo.
			Dentro alla cavità non c'è mai una differenza di potenziale diversa da zero, indipendentemente 
			dal potenziale a cui si trova il conduttore. Quindi il conduttore si comporta da \textbf{schermo} verso 
			il mondo esterno.
			Quindi possiamo dire che essendo la carica superficiale della cavità $\sigma_{sup.cav}$ uguale a zero, ricaviamo che
			\begin{enumerate}
				\item $E_{int.cond.} = 0 \Rightarrow \sum Q_{cav.sup} = 0$
				\item $\oint E = 0$ per il campo conservativo $\Rightarrow E_{cav} = 0$
			\end{enumerate}	
		\subsection{Condensatore}
			Il condensatore(noto come capacitore) e' un componente elettrico che immagazzina l'energia in un campo elettrostatico, 
			che crea una differenza di potenziale. Il condensatore e' un componente ideale che puo' mantenere la carica e 
			l'energia accumulata all'infinito.		 
		\subsection{Capacità}
			L'unita' di misura della capacità elettrica nel sistema SI e' il Farad (simbolo $F$) ed e' l'unita di 
			misura derivata dal rapporto tra il coulomb e il volt:
			\[ F = \frac{C}{V} = \frac{C^2}{J} \]
			Un conduttore con capacita' di un $1F$ varia di un volt il suo potenziale quando la sua carica immagazzinata varia di
			1 coulomb.
			\textbf{La capacita' di un consensatore} e' quindi uguale a \[ C = \frac{Q}{\vartriangle V} \]
		\subsection{Condensatori in Parallelo}
			I condensatori in parallelo sono condensatori che vengono collegati tra loro in modo tale che, ai loro estremi,
		 	sia presente la stessa differenza di potenziale $\triangle V$. 
		 	La carica elettrica totale e' data dalla somma delle cariche elettriche presenti su ciascun condensatore:
		 	\[ C_p = \sum_{i = 1}^n C_i \]
		\subsection{Condensatori in Serie}
			In un collegamento in serie, quando si conferisce una carica al sistema per induzione, 
			ogni condensatore presente sul circuito si carica di quella stessa carica, in moto tale da avere la stessa carica $Q$
			\[ \frac{1}{C_{s}} = \sum_{i = 1}^n \frac{1}{C_i} = \frac{Q}{C_{tot}}\]
			
	\newpage
	\section{Elettrostatica nei Dielettrici} 
		\subsection{Dipolo Elettrico}
			Un dipolo elettrico e' un sistema composto da due cariche elettriche uguali e opposte di segno e separate da una
			distanza costante nel tempo. Esso rappresenta l'approssimazione basilare del campo elettrico generato da un insieme 
			di cariche globalmente neutro.
		\subsection{Momento di Dipolo}
			\paragraph*{Descrizione}
			Dato un sistema di cariche, il momento elettrico, o momento di dipolo, e' una grandezza vettoriale che 
			quantifica la separazione tra le cariche positive e negative, ovvero la polarità del sistema, e si misura 
			in coulomb per metro.
			\paragraph*{Definizione}
			Date due cariche di segno opposto e uguale modulo $q$, il momento elettrico $p$ e' definito come :
			\[ p = qd = \sum_{i=1}^n q_i r_i \]
			dove $d$ e' il vettore spostamento dell'uno rispetto all'altro, orientato dalla carica negativa alla carica
			positiva e per il quale deve valere $d = 0$ e $r_i$ e' la posizione della carica $q_i$. \\
			Al momento di dipolo elettrico si associa un momento meccanico $M$ calcolato come \[ M = p \times E \]
		\subsection{Dipolo Elettrico in un campo elettrostatico}
			\subsubsection{Energia potenziale del dipolo}
			\[U(\overrightarrow{r}) = -\overrightarrow{p} \cdot \overrightarrow{E} \]
			\subsubsection{Forza}
			\[ \overrightarrow{F} = (\overrightarrow{p} \cdot \overrightarrow{\nabla}) \overrightarrow{E} \]
		\subsection{Dielettrici Lineari}
			\paragraph*{Descrizione}
			Un isolante dielettrico, detto anche dielettrico lineare, e' un materiale incapace di condurre la corrente 
			elettrica. Si tratta di materiali, privi di cariche libere, le quali non riescono a far fornire al campo elettrico 
			sufficiente energia per consentire gli elettroni di raggiungere la banda di conduzione.  Tuttavia il campo elettrico
			e' in grado di polarizzare il materiale, genera cioè lo spostamento delle cariche, elettroni e nuclei, 
			formando microscopici dipoli elettrici che generano all'interno del materiale un campo aggiuntivo opposto a 
			quello esterno, manifestando una proprietà detta dielettricità.
			\paragraph*{Definizione}
			La maggior parte dei materiali isolanti puo' essere trattata come un dielettrico lineare omogeneo ed isotropo. 
			Se considerassimo $D$ come un campo rispetto ai dielettrici, potremmo affermare che: 
			\[ D = \epsilon_0 E + P = \epsilon_0 ( 1 + \chi)E = \epsilon_r \epsilon_0 E = \epsilon E \]
			dove $\epsilon_0$ e' la costante elettrica del vuoto che vale $\frac{1}{c^2 \mu_0} 
			\cong 8,8541878176 \cdot 10^{-12} \frac{F}{m} $ \\
			Il numero $\epsilon_r$ e' la permittività elettrica relativa, mentre $\chi$ e' la suscettività 
			elettrica del mezzo, che e' definita come la costante di proporzionalità tra il campo $E$ ed il 
			conseguente vettore di polarizzazione $P$. Come conseguenza si ha che :
			\[ P = (\epsilon_r - 1 ) \epsilon_0 E = \epsilon_0 \cdot \chi \cdot E \]
			La suscettività e' quindi legata alla permittività relativa $\epsilon_r$ mediante la relazione 
			\[ \chi = \epsilon_r -1 \] che nel vuoto diventa $\chi = 0$.
		\subsection{Polarizzazione nel dielettrico}
			\begin{itemize}
				\item Dielettrico : $\vert \overrightarrow{P} \vert = \vert \sigma_p \vert$
				\item Dielettrico Lineare : $P = \epsilon_0 \chi E_{tot}$
				\item $E_{tot}$ dentro nel dielettrico : $E + \frac{P}{\epsilon_0} = \frac{\vert \sigma_l \vert}{\epsilon_0} = E_0$
			\end{itemize}
		\subsection{Teorema di Gauss nei Dielettrici}
			Nei dielettrici, le cariche elettriche libere sono le sorgenti del vettore induzione, mentre nel 
			vuoto lo erano per il campo elettrico.
			\[ 
				\Phi(\overrightarrow{E}) = \oint_{sup.chiusa.} \overrightarrow{E}d\overrightarrow{S} = \frac{Q_{tot}}{\epsilon_0} =
				Q_{polarizzata} + Q_{libera}
			\]
			\[ Q_{polarizzata} = -\sigma_p \sum\]con $\sum$ la superficie.
			\[ \oint PdS = P \sum \]
		\subsection{Vettore Induzione Dielettrica}
			Tramite il teorema di Gauss per i dielettrici possiamo affermare che il vettore spostamento $\overrightarrow{D}$ 
			e' uguale a :
			\[ \overrightarrow{D} = \epsilon_0 \overrightarrow{E} + \overrightarrow{P} \]
			Se considerassimo $D$ come un campo, allora avremmo la stessa formula descritta nel paragrafo \textbf{3.4}.
	
	\newpage
	\section{Energia Elettrostatica}
		\subsection{Energia di un sistema di cariche}
			Due cariche elettriche $Q$ e $q$ poste a una certa distanza $d$ posseggono una energia potenziale elettrica
			elettrica $U$, tale che \[ U = k \frac{Qq}{d} \]. Cosa succede se si avesse un sistema di $n$ cariche?
			Sappiamo che per tre cariche, ad esempio, si ha che :
			\[ U = k \frac{q_1 q_2}{d_{12}} + k \frac{q_1 q_3}{d_{13}} + k \frac{q_2 q_3}{d_{23}} \]
			con $d_{12}, d_{13} + d_{23}$ sono le distanze tra le cariche. Allora, seguendo lo stesso ragionamento 
			per $n$ cariche, avremo che :
			\[ U = k \cdot \sum_{i = 1}^n \sum_{j = i + 1}^n  \frac{q_i q_j}{d_{ij}} \]
		\subsection{Sistema di Conduttori}
			Consideriamo due conduttori carichi $A$ e $B$ in equilibrio e supponiamo che le linee di forza del campo
			elettrico $\overrightarrow{E}$ vadano da uno all'altro. E' possibile costruire una superficie chiusa $S$ 
			che intersechi i conduttori in corrispondenza delle superfici $dS_A$ e $dS_B$ e si chiuda all'interno di
			essi con le superfici $S_A$ e $S_B$. Il flusso del campo elettrico attraverso la superficie $S$ e' nullo
			siccome il campo e' nullo in corrispondenza delle superfici $S_A$ e $S_B$ ed e' parallelo alla superficie
			considerata nella regione spaziale compresa tra i conduttori. Pertanto dalla legge di Gauss segue che deve
			risultare nulla la somma delle cariche $dq_A$ e $dq_B$ interne ad $S$, localizzate sulle superfici $dS_A$ e
			$dS_B$ dei due conduttori in questo modo:
			\[ dq_A = -dq_B \]
			Si conclude affermando che le cariche su due elementi superficiali corrispondenti sono
			uguali in modulo ma di segno opposto. 
			Questo risultato prende il nome di \textbf{teorema degli elementi superficiali corrispondenti}
		\subsection{Energia immagazzinata in un condensatore}
			In generale, il lavoro e' espresso come prodotto della carica per la differenza di potenziale
			: $W = Q \cdot \triangle V$. In un condensatore le armature continueranno a caricarsi distribuendo la carica $Q$
			uniformemente su tutta la loro superficie fino a quando la differenza di potenziale $\triangle V$ tra le armature 
			non avrà raggiunto la differenza di potenziale della pila. E' possibile pero' suddividere il processo generale in tanti
			piccoli intervalli. Si può dimostrare che il lavoro complessivamente svolto durante il processo e' dato 
			dalla seguente formula:
			\[ W_c = \frac{1}{2}QV = \frac{1}{2}CV^2 = \frac{1}{2} \frac{Q^2}{C} \]
		\subsection{Energia del campo elettrico}
			Per l'energia elettrica si definisce densita volumica di energia il rapporto tra l'energia immagazzinata nel
			 condensatore e il suo volume interno:
			\[ W_{\overrightarrow{E}} = \frac{W_C}{S \cdot d}  = \frac{1}{2}\frac{Q^2}{C} \cdot \frac{1}{S \cdot d} 
			= \frac{1}{2}Q^2 \cdot \frac{d}{\epsilon \cdot S} \cdot \frac{1}{S \cdot d} = \frac{1}{2}Q^2 \cdot 
			\frac{1}{\epsilon \cdot S^2}\] 
			dove $S$ indica l'area delle armature, mentre $d$ la distanza che le separa. Nella formula possiamo isolare 
			il quadrato del rapporto $\frac{Q}{S}$ che rappresenta la densità superficiale di carica :
			\[ W_{\overrightarrow{E}} = \frac{1}{2\epsilon} \cdot \frac{Q^2}{S^2} = \frac{\sigma^2}{2\epsilon} \]
			Ora, moltiplicando il numeratore e il moltiplicatore per $\epsilon$ potremmo ottenere il rapporto 
			$\frac{\sigma}{\epsilon}$, che rappresenta appunto $E$
			\[ W_{\overrightarrow{E}} = \epsilon \cdot \frac{\sigma^2}{2\epsilon^2} = 
			\frac{\epsilon}{2} \cdot \frac{\sigma^2}{\epsilon^2} = \frac{\epsilon}{2} \cdot E^2 \]
		\subsection{Moto di una carica in un campo elettrico uniforme}
			In un campo elettrico $\overrightarrow{E}$ uniforme una carica $q$ risente di una forza
			$\overrightarrow{F} =q\overrightarrow{E}$ costante.
			\subsubsection{Moto della direzione del campo elettrico}
			Se una particella di carica $a$ e massa $m$ parte da ferma, oppure ha una velocità iniziale parallela alle linee del
			campo elettrico, il suo moto e' analogo a quello di un corpo soggetto alla forza peso, perché su di essa agisce una
			accelerazione costante che, per la seconda legge della dinamica, vale:
				\[ a = \frac{F}{m} = \frac{qE}{m} \]
			\subsubsection{Velocità finale con partenza da fermo}
			Una particella di carica $q$ e massa $m$ si trova in un campo elettrico uniforme con velocità iniziale nulla. 
			Essa si sposta dal punto iniziale $A$ a un punto finale $B$ posto sulla stessa linea di campo su cui si trova $A$.
			Se la particella si muove nel vuoto, su di essa non agiscono attriti. Per il teorema dell'energia cinetica abbiamo che
			\[ K_f = K_i + W_{A \to B} \] dove $K_i$ e' l'energia cinetica finale, $K_i$ e' quella iniziale e $W$ e' 
			il lavoro fatto dalla forza che agisce sulla particella. Visto che abbiamo $K_i = 0J$ (la particella parte da ferma) 
			e $K_f = \frac{1}{2} mv^2$, il teorema dell'energia cinetica diventa 
			\[ \frac{1}{2}mv^2 = W_{A \to B} \]
			Dalla definizione di differenza di potenziale si ottiene che $W_{A \to B} = q(V_A - V_B)$, per cui
			\[ \frac{1}{2} mv^2 = q(V_A - V_B) \]
			Da questa equazione finalmente otteniamo l'equazione finale, ovvero
			\[ v = \sqrt{\frac{2q (V_A - V_B)}{m}} \]
			\subsubsection{Moto parabolico}
			Consideriamo una carica $q$ di massa $m$  che entra tra due armature caricate di segni opposti, con 
			il vettore velocità $\overrightarrow{v_0}$ parallelo alle armature stesse.
			Una volta che si trova tra le armature, sulla carica agisce una forza $\overrightarrow{F} = q\overrightarrow{E}$,
			perpendicolare a $\overrightarrow{v_0}$. Per il secondo principio della dinamica, su di essa e' impressa una
			accelerazione costante $\overrightarrow{a} = \frac{q}{m}\overrightarrow{E}$, anch'essa perpendicolare a 
			$\overrightarrow{v_0}$. Cio' significa che la particella e' soggetto a due moti simultanei:
			\begin{itemize}
				\item di moto uniforme nella direzione e nel verso di $\overrightarrow{v_0}$
				\item di moto uniformemente accelerato nella direzione e nel verso di $\overrightarrow{a}$.
			\end{itemize}
			Se scegliamo un sistema cartesiano per rappresentare le formule, avremo che 
			\[ \begin{cases} x = v_0 t \\ y = \frac{1}{2} \frac{qE}{m}t^2 \end{cases}\]
			dove per $x$ abbiamo il moto uniforme orizzontale, 
			per $y$ il moto uniformemente accelerato verticale.
			
	\newpage
	\section{Correnti Elettriche}
		\subsection{Corrente elettrica}
			\paragraph*{Descrizione}			
			La corrente elettrica indica lo spostamento complessivo delle cariche elettriche, ovvero un qualsiasi moto ordinato
			definito operativamente come la quantità di carica elettrica che attraversa una determinata superficie nell'unità 
			di tempo.
			\paragraph*{Definizione}
			Si consideri un conduttore di sezione $S$ attraverso il quale vi sia un moto ordinato di cariche. Si definisce
			corrente elettrica la quantità di carica elettrica $\triangle Q$ che nell'intervallo di tempo $\triangle t$ attraversa 
			la superficie $S$: 
			\[ I = \frac{\triangle Q}{\vartriangle t} \]
			Essa e' una quantità scalare poiché non possiede una direzione. \\ La potenza $P$ sviluppata dal campo elettrico e'
			\[ P = \frac{dL}{dt} = I \triangle V \]
		\subsection{Forza Elettromotrice}
			La forza elettromotrice, comunemente abbreviata in \textbf{f.e.m.} e' il rapporto tra il lavoro compiuto da un
			generatore elettrico per muovere le cariche dal polo a basso potenziale al polo a potenziale più alto 
			e l'unita di carica spostata.
			\subsubsection{F.e.m. di un generatore}
				Il lavoro $L$ necessario al trasporto delle cariche verso i rispettivi poli e' direttamente proporzionale
				alla quantità di carica $q$; la forza elettromotrice $E$ e' definita come quantità di lavoro compiuto per unità
				di carica, secondo la formula:
				\[ E = \frac{dL}{dq} \hspace{0.5cm} [V] \]
				In un circuito chiuso, la differenza di potenziale $\triangle V$ misurata ai poli di un generatore reale risulta 
				sempre leggermente inferiore alla forza elettromotrice del generatore per effetto della resistenza interna
				$r_i$ dello stesso:
				\[ \triangle V = E - i r_i \]
		\subsection{Modello classico della conduzione elettrica}
			L'Intensità di corrente elettrica può essere espressa in termini di quantità microscopiche quali la velocità
			di deriva $v_D$ degli elettroni liberi all'interno del metallo (ovvero il valore medio della velocità degli
			elettroni lungo la direzione del campo), il numero di elettroni liberi $n_e$ per unità di volume del metallo e 
			la carica elettrica $e$ dell'elettrone. Esprimendo $I$ in funzione di tali quantità si ha che:
			\[ I = n_e \cdot e \cdot A \cdot v_D \] che confrontata con $I = \frac{1}{\rho} EA $ si ottiene
			\[ \rho = \frac{E}{n_e e v_D} \] con $\rho$ la densità volumetrica di carica, ovvero $\rho = \frac{Q}{V}$.\\
			Poiché $\rho$ e' indipendente dal campo elettrico, la velocità di deriva degli elettroni $v_D$ risulta costante e
			direttamente proporzionale all'intensità $E$ del campo elettrico. \\
			Consideriamo dunque un elettrone scelto a caso e indichiamo con $\tau$ il tempo medio trascorso dall'ultima 
			collisione dell'elettrone con il reticolo. Poiché l'elettrone e' soggetto ad una accelerazione 
			$a = e \frac{E}{m_e}$, con $m_e$ la massa dell'elettrone, la velocità di deriva può essere espressa come : 
			\[ v_D = \frac{e E}{m_e} \tau  = - \frac{e}{m_e}\tau \overrightarrow{E} \]
			Inoltre, possiamo affermare che (tramite la sostituzione) 
			\[ \rho = \frac{m_e}{n_e e^2 \tau} \]
			Se $v_m$ indica la velocità media degli elettroni, la distanza media percorsa da un elettrone tra due 
			urti consecutivi e' \[ \Lambda = v_m \tau \]
			che sostituita nella equazione precedente permette di esprimere la resistiva del metallo in funzione 
			di $\Lambda$ e $v_m$, cioè:
			\[ \rho = \frac{m_e v_m}{n_e e^2 \Lambda} \]
			con $\Lambda$ il cammino libero medio. In accordo alle leggi di Ohm la resistività del metallo e'
			indipendente dal campo elettrico 
		\subsection{Equazione di continuità per la carica}
			\paragraph*{Descrizione}
			La legge di conservazione della carica elettrica e' l'equazione di continuità per la carica elettrica. 
			La legge afferma che il flusso della densità di corrente elettrica attraverso una qualunque superficie 
			chiusa e' pari alla variazione della carica elettrica situata nel volume racchiuso dalla superficie.
			\paragraph*{Definizione}
			La legge di conservazione per la carica elettrica in forma differenziale e' l'equazione di continuità:
			\[ \frac{\partial p}{\partial t} + \nabla \cdot J = 0 \] dove $J = \rho v$ e' la densità di corrente elettrica, 
			con $\rho$ la densità di carica elettrica e $v$ la velocità di deriva. Dopo aver integrato in un volume ed essersi 
			serviti del teorema della divergenza si ottiene la forma integrale:
			\[ \frac{d}{dt} \int_{v} \rho d r^3 + \oint_{\partial v} J \cdot dS = 0 \] che può essere scritta come
			\[ \frac{dQ}{dt} + I = 0 \] dove $Q$ e' la carica elettrica contenuta nel volume di integrazione e $I$ e' 
			la corrente elettrica netta uscente dalla superficie che lo delimita. 
		\subsection{Leggi di Ohm}
			\subsubsection{Prima Legge di Ohm}
				In un filo conduttore l'intensità di corrente $I$ e' direttamente proporzionale al voltaggio $V$ ed 
				inversamente proporzionale alla resistenza $R$. La formula della prima legge di Ohm e' la seguente: 
				\[ I = \frac{V}{R} \hspace{1.0cm} V = I \cdot R \hspace{1.0cm} R = \frac{V}{I} \]
			\subsubsection{Seconda Legge di Ohm}
				La resistenza di un filo conduttore e' direttamente proporzionale alla sua lunghezza $l$ ed 
				inversamente proporzionale alla sua sezione $s$. La resistenza dipende inoltre dalla natura del 
				materiale: ogni materiale ha la sua \textbf{resistenza specifica} che si indica con la lettera $\rho$
				\[ R = \rho \frac{l}{s} \]
		\subsection{Effetto Joule}
			L'effetto Joule e' il fenomeno per cui il passaggio di corrente elettrica, facendo resistenza con il conduttore, 
			produce calore. L'effetto Joule e' il principio di funzionamento della lampadina a incandescenza; difatti il 
			calore dovuto a questo fenomeno provoca l'incandescenza del filamento di tungsteno
		\subsection{Resistore}
			Il resistore, anche chiamato resistenza, e' un tipo di componente elettrico destinato a opporre una 
			specifica resistenza elettrica al passaggio della corrente elettrica.
			\subsubsection{Resistore ideale}
				Nella teoria dei circuiti il resistore e' un componente ideale che risponde, se lineare, alla legge di Ohm. 
				La sua equazione caratteristica si desume dalla legge di Ohm, ovvero 
				\[ V = R \cdot I \]
				Oppure 
				\[I = G \cdot V \] dove $G$ corrisponde alla conduttanza. Queste due formule si scelgono a seconda che 
				si consideri come parametro la resistenza o la conduttanza.  La potenza assorbita dal resistore e' data da:
				\[ P = V \cdot I = R \cdot I^2 = \frac{V^2}{R} = G \cdot V^2 \]
			\subsubsection{Resistore reale}
				I resistori reali sono caratterizzati dal valore della loro resistenza elettrica, espressa in ohm ($\omega$), 
				nonché dalla massima potenza (ovvero energia per unità di tempo) che possono dissipare, senza distruggersi, 
				espressa in watt.
		\subsection{Leggi di Kirchoff}
			\subsubsection{Prima legge di Kirchoff(ai nodi)}
				La somma delle correnti entranti in un nodo e' uguale alla somma delle correnti uscenti, ovvero la somma 
				algebrica delle correnti che interessano un nodo e' uguale a zero, tale che :
				\[ {\sum}_i I_i = 0 \]
			\subsubsection{Seconda legge di Kirchoff (alle maglie)}
				La somma algebrica delle forze elettromotrici (f.e.m. : i generatori) e delle cadute di tensione
				(le differenze di potenziale ai capi di ogni singola resistenza) che si incontrano in una maglia e' uguale
				a zero, tale che 
				\[ {\sum}_i E_i + {\sum}_i V_i = 0 \]
				Inoltre
				\[ I = \frac{V_{AB} + E}{R} \]
		\subsection{Circuito elementare}
			Il circuito elettrico elementare e' costituito da un generatore da un utilizzatore, dai conduttori in collegamento e da 
			un interruttore. Il generatore ha la funzione di trasformare energia non elettrica in energia elettrica, 
			l'utilizzatore di trasformare energia elettrica in altre forme di energia.
			\subsubsection{Il generatore elettrico}
				Analizzando un generatore di corrente continua possiamo dire che esso e' caratterizzato da una forza 
				elettromotrice (f.e.m.) e da una resistenza interna. La differenza di potenziale ai suoi morsetti differisce dalla
				forza elettromotrice per la caduta di tensione che si manifesta nella resistenza interna per effetto della 
				corrente erogata. Si può affermare che la differenza di potenziale (d.d.p), solitamente indicata con $U$,
				e' presente anche se il generatore non e' collegato a nessun utilizzatore. La forza elettromotrice si indica con
				la lettera $E$ e rappresenta una caratteristica propria di ogni generatore. Parte di essa viene persa internamente 
				al generatore stesso poiché essendo realizzato con materiali non perfettamente conduttori, presenta una sua 
				resistenza interna $r$.  
		\subsection{Carica di un Condensatore}
			Un condensatore e' formato da due lastre metalliche parallele (dette armature), poste ad una distanza $d$ l'una 
			dall'altra. Inizialmente le armature sono messe a terra, ossia possiedono carica nulla, e sono neutre. \\
			Per caricare un condensatore si comincia da una sola armatura, che viene collegata ad un generatore di corrente. 
			Il generatore di tensione sottopone l'armatura ad una certa differenza di potenziale $\triangle V$, cosicché 
			la carica finale registrata sulla piastra sarà data da \[ Q = C \cdot \triangle V \]
			Tuttavia la carica $Q$ non viene raggiunta immediatamente, ma solo dopo un certo intervallo di tempo; mentre
			inizialmente si ha una intensità di corrente particolarmente elevata, con l'aumentare della carica sull'armatura,
			aumenta anche la forza di repulsione tra le cariche dello stesso segno, e di conseguenza aumenta il lavoro che deve 
			compiere il generatore contro le forze elettriche repulsive. Il flusso delle cariche elettriche sull'armatura
			diventa sempre più lento, fino ad esaurirsi completamente al raggiungimento del valore massimo, ossia della 
			carica $Q$. E' possibile esprimere la carica istantanea presente sull'armatura del condensatore, in funzione 
			del tempo, tramite la seguente formula: \[q(t) = Q(1 - e^{- \frac{t}{RC}}) \] dove Q indica la carica massima
			raggiunta, $R$ la resistenza elettrica del circuito e $C$ la capacità del condensatore. E' possibile determinare anche
			l'intensità di corrente istante per istante, anch'essa avente andamento esponenziale ed e' data dalla formula:
			\[ i = \frac{f_{em}}{R} \cdot e^{-\frac{t}{RC}} \] con $f_{em}$ la forza elettromotrice.
		\subsection{Scarica di un Condensatore}
			Anche nel caso del processo di scarica del condensatore, la carica $Q$ raggiunta \textbf{non} viene eliminata
			tutta istantaneamente. La legge che descrive la quantità di carica presente sull'armatura in funzione del tempo e' 
			data dalla formula:
			\[ q(t) = Q \cdot e^{-\frac{t}{RC}} \]
			L'Intensità di corrente durante questo processo ha la stessa espressione di quella precedente: infatti, anche in 
			questo caso, inizialmente si ha una corrente piuttosto intensa, che pero diminuisce di intensità man 
			mano che le cariche elettriche sulle armature diminuiscono:
			\[ i = \frac{f_{em}}{R} \cdot e^{-\frac{t}{RC}} \]
	
	\newpage
	\section{Magnetostatica nel Vuoto}
		\subsection{Campo Magnetico}
			Il campo magnetico e' un campo vettoriale solenoidale generato nello spazio dal moto di una carica elettrica o da un
			campo elettrico variabile nel tempo. Insieme al campo elettrico esso costituisce il campo elettromagnetico, 
			responsabile dell'interazione elettromagnetica. Solitamente viene indicato con $B$.
		\subsection{Forza di Lorentz}
			Sia data una carica elettrica puntiforme $q$ in moto con velocità istantanea $v$ in una regione caratterizzata
			dalla presenza di un campo elettrico $E$ e un campo magnetico $B$. La forza di Lorentz e' la forza $F$ esercitata
			dal campo elettromagnetico sulla carica, ed e' proporzionale a $q$ e al prodotto vettoriale tra $v$ e $B$ secondo la 
			relazione:
			\[ F(r,t,q) = q[(E(r,t) + v \times B(r,t)] \] dove $r$ e' la posizione della carica, 
			$v = \dot{r}$ la sua velocità e $t$ il tempo. Una carica positiva viene accelerata nella direzione di $E$ e viene 
			curvata nella direzione perpendicolare al piano formato da $v$ e $B$.
			Nel caso in cui sia presente \textbf{solo il campo magnetico}, la formula puo' essere applicata al caso di un 
			circuito filiforme di lunghezza $l$ percorso dalla corrente elettrica $I$ :
			\[ F = I \int_l dl \times B \] e sapendo che per definizione \[ I dl = J dv \] con $J$ la densità di corrente,
			si può estendere al caso più generale di un volume $V$ percorso da una corrente descritta dalla densità 
			di corrente, per il quale si ha 
			\[ F = I \int_l dl \times B = \int_V J \times B dv \]
			Dal momento che la forza di Lorentz e' legata al campo magnetico tramite il prodotto vettoriale, la forza e 
			il campo non hanno la stessa direzione, essendo perpendicolari. Come conseguenza di ciò, 
			la forza di Lorentz non compie lavoro.
		\subsection{Seconda legge di Laplace}
			Consideriamo un filo percorso da una corrente $I$ lungo $l$: se la direzione del filo e' ortogonale alla direzione del
			campo magnetico sperimentalmente si trova che la forza che agisce su di esso sarà mutuamente perpendicolare sia alla
			direzione del campo magnetico che alla direzione del filo e proporzionale all'intensità della corrente 
			ed alla lunghezza del filo. Consideriamo inoltre un generatore di $f.e.m.$ che fa scorrere la corrente nel 
			circuito. Nel piano normale al filo vi e' un campo magnetico di intensità $\vert B \vert$. Si può definire
			il campo di induzione magnetica, a partire dalla forza $\overrightarrow{F}$ misurabile come:
			\[ \vert B \vert = \frac{\vert F \vert}{I \vert dl \vert} \] 
			Se il filo e' parallelo alle linee del campo su di esso in generale non agisce nessuna forza. L'espressione 
			matematica della forza di un campo magnetico su un filo percorso da corrente viene detta seconda legge di Laplace:
			\[ d\overrightarrow{F} = I \overrightarrow{dl} \times \overrightarrow{B} \]
			\textbf{Nota}: La famosa \textbf{regola della mano destra} puo' essere d'aiuto nel calcolo della forza agente. 
			Infatti, se la direzione della corrente e' quella dell'indice della mano destra e quella del campo magnetico e' 
			il medio, la direzione forza e' data dalla direzione del pollice
		\subsection{Moto di una carica in un campo magnetico}
			Una carica puntiforme che si muove con velocità $v$ all'interno di un campo magnetico e' sottoposta alla forza
			di Lorentz. Tale forza, agendo sulla carica, ne modifica la traiettoria, costringendola a seguire un moto preciso.
			Si può affermare che la forza di Lorentz che agisce su una particella non modifica il modulo della sua velocità, ma
			soltanto la sua traiettoria: ciò che cambia e' la \textbf{direzione} e il \textbf{verso} della velocità 
			della particella. Un moto uniforme che riflette queste stesse caratteristiche e' quello circolare, in cui la 
			forza centripeta, rivolta cioè verso il centro della circonferenza, e perpendicolare in ogni punto alla velocità
			tangenziale. Conoscendo l'espressione della forza di Lorentz, e quella della forza centripeta, si riesce a determinare
			ulteriori informazioni sul moto della particella, come il raggio della circonferenza:
			\[ F_q = q \cdot v \cdot B  \hspace{1.0cm} F_c = m \cdot \frac{v^2}{r} \rightarrow r = \frac{m \cdot v}{q \cdot B} \]
			dove $m$ indica la massa della particella, $v$ la sua velocità, $q$ la sua carica e $B$ il campo magnetico cui essa
			e' sottoposta. La velocità della particella, in un moto circolare uniforme, e' descritta come:
			\[ v = \frac{2\pi r}{T} \]
			Sostituendo questa formula nell'espressione precedente, possiamo esprimere anche il periodo del moto circolare:
			\[ r = \frac{m \cdot v}{q \cdot B} = \frac{m}{q \cdot B} \cdot \frac{2\pi r}{T}
			\rightarrow T = \frac{2\pi m}{q \cdot B} \]
		\subsection{Effetto Hall}
			L'effetto Hall e' la formazione di una differenza di potenziale, detto potenziale di Hall, sulle facce opposte 
			di un conduttore elettrico dovuta a un campo magnetico perpendicolare alla corrente elettrica che scorre in esso.
			L'elemento di Hall e' formato da una striscia di materiale che puo' condurre elettricità, di solito un metallo
			conduttore o un semiconduttore. Come forma fisica si usa una striscia perché ha due dimensioni, lo spessore e'
		    trascurabile rispetto alle altre due. In questo materiale viene fatta scorrere una corrente applicando una batteria 
		    ai suoi capi. Nei conduttori gli elettroni si muovo dal polo negativo a quello positivo della batteria. Il magnete 
		    crea un campo magnetico che va dal polo Nord al polo Sud dello stesso magnete. L'elemento di Hall e' immerso in 
		    questo campo magnetico. Gli elettroni di conduzione si muovo, hanno cioè velocità, e risentono del campo 
		    magnetico: su di loro agisce la forza di Lorentz:
		    \[ \overrightarrow{F} = q\overrightarrow{v} \times \overrightarrow{B}\] dove $q$ e' la carica dell'elettrone pari a
		    circa $-1,6022 \times 10^{-19} C$, $\overrightarrow{v}$ e' la velocità (di deriva) dell'elettrone 
		    e $\overrightarrow{B}$ e' il campo magnetico. Man mano che gli elettroni si muovo, gli accumuli di carica aumentano.
		    Dopo un tempo abbastanza lungo, si arriverà' ad una condizione di equilibrio dinamico delle forze 
		    fra il campo elettrico e la forza di Lorentz, ovvero: \[ qE = qv_dB \] dove $E$ rappresenta il modulo del campo
		    elettrico tra i due estremi del conduttore. Sia la tensione di Hall 
		    \[triangle V_H = Ed = v_d Bd = \frac{1}{nq}\frac{iB}{d} \] troviamo $E$ e quindi la velocità' delle cariche elettriche.
		    Il rapporto \[ R_H = \frac{1}{nq} \] viene detto \textbf{costante di Hall}. Tramite la costante di Hall, si può
		    determinare il numero di cariche elettriche che attraversano una sezione dell'elemento di Hall, ovvero 
		    la \textbf{corrente di Hall}:
		    \[ n = \frac{i}{nv_d A}\] dove $n$ e' il numero di cariche per unita di volume, $i$ e' l'intensità di corrente 
		    e $A$ e' l'area dell'elemento di Hall.
		\subsection{Dipolo Magnetico}
			Un dipolo magnetico e' un magnete ottenuto considerando una spira microscopica percorsa da corrente elettrica. 
			La grandezza che caratterizza principalmente un dipolo magnetico e' il momento magnetico, 
			che quantifica la tendenza del dipolo a orientarsi in una data direzione presenza di un campo magnetico esterno.
		\subsection{Dipolo Magnetico in campo esterno}
			Il campo magnetico generato da un dipolo e' calcolato considerando una spira percorsa da corrente elettrica. 
			Nel limite in cui le sue dimensioni diminuiscono mantenendo costante il prodotto tra corrente ed area si ottiene 
			il modello per il dipolo magnetico. Il potenziale magnetico della spira e' dato dall'espressione
			\[ A = \frac{\mu_0}{4\pi} \frac{m \times r}{r^3} \]
			dove $m$ e' il momento di dipolo magnetico e $\mu_0$ e' la permeabilità magnetica del vuoto. 
			L'intensità del campo magnetico $B$ e' data da 
			\[B(r) = \nabla \times A = \frac{\mu_0}{4\pi} \bigl( \frac{3r(m \cdot r)}{r^5} - \frac{m}{r^3} \bigr) \]
			Si ottiene in questo modo il potenziale magnetico scalare: 
			\[ \psi (r) = \frac{m \cdot r}{4\pi r^3} \]
			da cui si ha che l'intensità di $H$ e' 
			\[ H(r) = -\nabla \psi = \frac{1}{4\pi} \bigl( \frac{3r ( m\cdot r) }{r^5} - \frac{m}{r^3} \bigr) = \frac{B}{\mu_0} \]
			Tale campo e' simmetrico rispetto alle rotazioni intorno all'asse del momento magnetico.
		\subsection{Campi magnetici generati da correnti}
			Consideriamo un filo qualunque percorso da corrente $i$ ed individuiamo un elemento $ds$ (tangente al filo ed 
			orientato secondo la corrente). Possiamo trovare che il contributo di campo magnetico generato da questo elemento
			e' \[dB = \frac{\mu_0}{4\pi} \frac{i \cdot ds \cdot sin\theta}{r^2} \] con $\theta$ l'angolo tra $ds$ e il vettore
			$\overrightarrow{r}$ che individua il punto nello spazio nel quale si calcola $dB$. La quantità $\mu_0$ e' 
			detta permeabilità magnetica nel vuoto ed ha valore 
			\[ \mu_0 = 4\pi \cdot 10^{-7} \frac{T \cdot m}{A} \] La direzione esatta si ricava dal prodotto vettoriale 
			$d\overrightarrow{S} \times \overrightarrow{u_r}$, per cui l'espressione precedente la si può scrivere come
			\[ d\overrightarrow{B} = \frac{\mu_0}{4\pi} \frac{i \cdot d\overrightarrow{s} \times \overrightarrow{u_r}}{r^2} \]
			nota come \textbf{prima legge elementare di Laplace}. Per un circuito finito si avrà
			\[ \overrightarrow{B} = \frac{\mu_0 i}{4\pi} \oint {\frac{\overrightarrow{ds} \times \overrightarrow{u_r}}{r^2}} \]
			che viene detta \textbf{Legge di Ampere-Laplace}.
		\subsection{Legge di Ampere}
			Il teorema di Ampere e' una legge fisica che afferma che l'integrale lungo una linea chiusa
			del campo magnetico e' uguale alla somma delle correnti elettrica a essa concatenate moltiplicata per la costante di
			permeabilità magnetica del vuoto. \\
			La legge di Ampere può essere espressa sia in termini del campo magnetico nel vuoto $B$ , sia in termini del campo 
			magnetico nei materiali $H$. Nel secondo caso gli effetti di polarizzazione magnetica sono compresi nella 
			definizione di $ H = \frac{B}{\mu_0} - M $. La legge afferma che l'integrale lungo una linea chiusa $\partial S$ del 
			campo magnetico $B$ e' uguale alla somma algebrica delle correnti elettrice $I_i$, concatenate a $\partial S$ 
			moltiplicata per la costante di permeabilità magnetica del vuoto $\mu_0$:
			\[ \oint_{\partial S} B \cdot dr = \mu_0 \sum_i I_i = \mu_0 I \] In termini della corrente $I'$ relativa a $H$ si ha:
			\[ \oint_{\partial S} H \cdot dr = I' \]
			Poiché la corrente netta che passa attraverso le superfici $S$ delimitate dalla curva chiusa $\partial S$ e' il
			flusso di una densità di corrente elettrica $j = pv$, in cui $v$ e' la velocità delle cariche che compongono
			la corrente e $p$ la loro densità volumica, la legge di Ampere si scrive:
			\[ \oint_{\partial S} B \cdot dr = \mu_0 \int_S J \cdot dS = \mu_0 I \]
			\[ \oint_{\partial S} H \cdot dr = \int_S J' \cdot dS = I' \]
			Utilizzando il teorema del rotore:
			\[ \oint_{\partial S} H \cdot dr = \int_S \nabla \times H \cdot dS = \int_S J \cdot dS \] eguagliando gli integrandi
			si ottiene la forma locale della legge di Ampere:
			\[ \nabla \times H = J \] che costituisce la quarta equazione di Maxwell solo nel caso stazionario.\\
			\textbf{Nel caso si abbia il caso non stazionario}, la legge di Ampere mostra come un campo elettrico
			variabile nel tempo sia sorgente di un campo magnetico. Inserendo la prima legge di Maxwell
			nell'equazione di continuità si ottiene: \[ 0 = \nabla \cdot J + \frac{\partial p}{\partial t} = 
			\nabla \cdot \bigl ( J + \epsilon_0 \frac{\partial E}{\partial t}  \bigr ) \] dove il termine 
			\[ J_s = \epsilon_0 \frac{\partial E}{\partial t} \] e' detto \textbf{densità di corrente di spostamento}, 
			e si somma alla densità di corrente nel caso non stazionario. Inserendo la densità di corrente 
			generalizzata si ottiene la quarta equazione di Maxwell nel vuoto:
			\[ \nabla \times B = \mu_0 ( J + J_s) \]
			Nel caso in cui non ci si trovi piu nel vuoto, la legge di Ampere-Maxwell assume la forma piu generale:
			\[ \nabla \times H = J + \frac{\partial D }{\partial t} \]
		\subsection{Discontinuità del campo B}
			Si consideri una lamina nel piano $xyz$ percorsa da una corrente superficiale omogenea, tale che 
			\[ \overrightarrow{J} = J \overrightarrow{e_x} \] Lo spazio e' diviso in 2 regioni, in modo tale da considerare
			$z > 0$ e $z < 0$. La discontinuità della componente tangenziale del campo magnetico si calcola in questo modo:
			\[ \vartriangle \overrightarrow{B} = \overrightarrow{B}(z_+) - \overrightarrow{B}(z_-) = \mu_0 J \overrightarrow{e_y}\]
			Nel caso di una distribuzione superficiale di carica elettrica la discontinuità e' nella componente normale del 
			campo elettrico.
		\subsection{Potenziale Vettore}
			\paragraph*{Descrizione} Il potenziale vettore e' un campo vettoriale il cui rotore e' un dato campo vettoriale. 
			E' l'analogo del potenziale scalare, che e' un campo scalare il cui gradiente e' un dato campo vettoriale.
			\paragraph*{Definizione} Dato un campo vettoriale $ \alpha : \Omega_2 \subseteq \mathbb{R}^3 \rightarrow \mathbb{R}^3$,
			il potenziale vettore di $\alpha$ e' un campo $\beta_2 : \Omega_2 \subseteq \mathbb{R}^3 \rightarrow R^3$ definito
			formalmente dalla relazione 
			\[ \alpha = \nabla \times \beta_2 \] ovvero $\alpha$ e' il rotore di $\beta_2$. Poiché la divergenza di un rotore
			e' nulla, $\alpha$ deve avere divergenza(campo scalare) nulla, ovvero 
			\[ \nabla \cdot \alpha = 0 \]
			Esplicando le componenti del rotore di $\beta_2$, si ottiene il seguente sistema di 4 funzioni a tre variabili, con 
			9 gradi di libertà:
			\[ \begin{cases} 
					\frac{\partial \beta_{2z}}{\partial y} - \frac{\partial \beta_{2y}}{\partial z} = \alpha_x \\
					\frac{\partial \beta_{2x}}{\partial z} - \frac{\partial \beta_{2z}}{\partial x} = \alpha_y \\
					\frac{\partial \beta_{2y}}{\partial x} - \frac{\partial \beta_{2x}}{\partial y} = \alpha_z 
			\end{cases} \]
		\subsection{Campo Solenoidale}
			Nel calcolo vettoriale un campo vettoriale $V$ continuo in un insieme aperto $A \subset \mathbb{R}^3$ si definisce 
			solenoidale se il flusso attraversa una qualsiasi superficie chiusa $S \subseteq A$ e' nullo:
			\[ \int_S V \cdot \widehat{n} d\sigma =0 \]. Si può affermare che il campo vettoriale $V$ e' solenoidale se 
			il flusso di $V$ attraversa una qualsiasi superficie $S \subseteq A$ dipende solo dal bordo della superficie.
			Una proprietà del campo vettoriale solenoidale e' quella di avere le linee di campo chiuse. Inoltre, poiche 
			la divergenza del campo e' nulla, e' possibile definire un potenziale vettore $A$, il cui rotore sia appunto
			il campo :
			\[ \nabla \times A = V \]. 
		\subsection{Flusso Magnetico}
			\paragraph*{Descrizione} Il flusso magnetico e' il flusso del campo magnetico attraverso una superficie. Essa e' 
			una grandezza scalare che dipende dall'angolo d'incidenza delle linee di campo, dal valore della permeabilità
			magnetica e dall'area della superficie stessa. Viene spesso indicato nelle formule con $\Phi_B$.
			\paragraph*{Definizione}
			Immaginando di scomporre una superficie generica $S$ in tessere di area infinitesimale $dS$, e' possibile determinare 
			per ciascuna di queste la componente $B_\bot$ del vettore di campo magnetico $B$ perpendicolare alla superficie 
			nella posizione della tessera considerata. Risulta dunque \[ B_\bot = B cos \alpha \], dove $\alpha$ indica l'angolo
			compreso fra $B$ e la normale alla superficie. In generale, questa componente varia per ogni punto della 
			superficie. Il flusso magnetico $d \Phi_B$ che attraversa questa area e' infinitesimale e' : \[ d \Phi_B = B \cdot dS \]
			Il flusso magnetico complessivo attraverso l'intera superficie e' l'integrale dei contributi delel componenti
			infinitesimali $d\Phi_B$ di tutti i punti della superficie:
			\[ \Phi_B = \int_S d\Phi_B = \oint_{\partial S} A \cdot d\ell \] con $A$ il potenziale vettore e $\partial S$
			il contorno di $S$. Per un campo magnetico omogeneo, l'integrale si semplifica e si ottiene il prodotto scalare tra 
			$B$ e $S$:
			\[ \Phi_B = B \cdot S = B_\bot S = BS cos \alpha \]
			essendo $\alpha$ l'angolo compreso fra $B$ e $S$    
	\newpage
	\section{Campi Variabili Nel Tempo}
		\subsection{Induzione Elettromagnetica}
			L'induzione elettromagnetica e' il fenomeno fisico che lega l'elettricità e il magnetismo tra loro, 
			visto che una corrente elettrica genera un campo magnetico e una variazione del campo magnetico 
			genera una corrente elettrica in un conduttore. 
		\subsection{Teorema del Flusso}
			\paragraph*{Definizione}
				Sia $F : \mathbb{R}^3 \smallsetminus \lbrace 0 \rbrace \rightarrow \mathbb{R}^3$ un campo vettoriale definito come
				\[F = F_1 \frac{r}{r^3} \]. Data una superficie chiusa $\partial V$ che contenga l'origine e tale che ogni
				 semiretta uscente dall'origine intersechi la superficie una e una sola volta, il teorema del flusso afferma che:
				 \[ \Phi_{\partial V} (F) = 4 \pi F_1 \] dove $\Phi_{\partial V} (F)$ e' il flusso di $F$ sotto l'angolo
				 solido giro $4\pi$. La relazione differenziale equivale a :
				 \[ \nabla \cdot F = \sigma (r) \]
			\paragraph*{Applicazione nel campo elettrico}
				Il campo elettrico nel punto $r$ generato da una carica totale $Q_V$ posta nel punto $r_0$ vale
				\[ D = \frac{Q_V}{4\pi}\frac{r - r_0}{\vert r - r_0 \vert^3} \]
				Si ottiene che il flusso attraversa il bordo $\partial V$ di un volume $V$ e' dato da:
				\[ \Phi_{\partial V}(D) = Q_V \] mentre se la superficie $\partial V$ non contiene $r_0$ il flusso e' nullo. Nel
				caso di più cariche puntiformi interne alla superficie :
				\[ \Phi_{\partial V}(D) = \sum_k q_k = \int_V p dv \]
				Grazie al teorema della divergenza, si ottiene che \[ \nabla \cdot D = p\] con $p$ la densità delle cariche libere.
				Il teorema assume quindi le seguenti formulazioni globale e locale:
				\[ \Phi_{\partial V}(E) = \frac{Q_V}{\epsilon}  \hspace{1.0cm} \nabla \cdot E = \frac{p}{\epsilon} \] 
				con $\epsilon = \epsilon_0 \cdot \epsilon_r$.
		\subsection{Legge di Faraday}
			La legge di Faraday descrive il manifestarsi di due fenomeni distinti: la forza elettromotrice dovuto alla 
			forza di Lorentz che si manifesta a causa del moto di una spira in un campo magnetico, 
			e la forza elettromotrice causata dal campo elettrico generato dalla variazione di flusso del campo magnetico.
			\subsubsection{Forma Globale}
				La legge di Faraday afferma che la forza elettromotrice $\vartriangle V$ indotta da un campo magnetico $B$
				in una linea chiusa $\partial \sum$ e' pari all'opposto della variazione nell'unita' di tempo del flusso
				magnetico $\Phi_{\sum} (B)$ del campo attraverso la superficie $\sum (t)$ che ha quella linea come frontiera:
				\[ \triangle V = - \frac{\partial \Phi_{\sum} (B)}{\partial t} \] dove il flusso magnetico 
				e' dato dall'integrale
				di superficie :
				\[ \Phi_{\sum} = \int_{\sum (t)} B(r,t) \cdot dA \] con $dA$ elemento dell'area $\sum$ attraverso la quale viene 
				calcolato il flusso. La forza elettromotrice e' definita mediante il lavoro svolto dal campo elettrico per unita'
				di carica $q$ del circuito:
				\[ \triangle V = \frac{1}{q} \oint_{\partial \sum} F \cdot d\ell = \oint_{\partial \sum} (E + v \times B) \cdot 
				d \ell \] dove $\partial \sum$ e' il bordo di $\sum$ e $F = q(E + v \times B)$ e' la forza di Lorentz. In caso di
				circuito stazionario, l'integrale assume la forma :
				\[ \oint_{\partial \sum } E \cdot d\ell = -\frac{\partial \Phi_{\sum} (B)}{\partial t} \]
			\subsubsection{Forma Locale}
				La forma locale (o differenziale) della legge di Faraday e' legata alla forma globale dal teorema del rotore:
				\[ \oint_{\partial S} E \cdot dr = \int_S \nabla \times E \cdot ds \]
				Per la definizione di flusso magnetico, e poiché il dominio di integrazione e' supposto costante nel tempo, si ha:
				\[ -\frac{\partial \Phi_{S} (B)}{\partial t} = -\frac{\partial}{\partial t} \int_S B \cdot ds = \int_S -
				\frac{\partial B}{\partial t} \cdot ds \]. Uguagliando gli integrandi segue la forma locale della legge di Faraday,
				che rappresenta la terza equazione di Maxwell:
				\[ \nabla \times E = -\frac{\partial B}{\partial t} \]
		\subsection{Legge di Lenz}
			La legge di Lenz e' una conseguenza del terzo principio della dinamica e della legge di conservazione dell'energia che 
			determina la direzione della forza elettromotrice risultante dall'induzione elettromagnetica in un circuito elettrico.
			La legge stabilisce che la variazione temporale del flusso del campo magnetico attraverso l'area abbracciata 
			da un circuito genera nel circuito una forza elettromotrice che contrasta la variazione. La legge fornisce una
			caratterizzazione qualitativa della direzione della corrente indotta nel circuito, e si esprime attraverso 
			il segno meno della legge di Faraday:
			\[ \epsilon = -\frac{\partial \Phi (B)}{\partial t} \] dove $\Phi$ e' il flusso del campo magnetico concatenato ed 
			$\epsilon$ e' la forza elettromotrice.
		\subsection{Bilancio Energetico}
			Il bilancio energetico complessivo di un dispositivo capace di trasformare energia elettrica in energia meccanica e/o
			viceversa deve tenere conto, oltre che dell'energia elettrica, dell'energia associata al campo elettromagnetico e
			dell'energia meccanica, anche delle perdite di energia. Il bilancio energetico nell'unita di tempo si scrive come 
			segue: \begin{center} (Energia Elettrica) = (Perdite Elettriche) + (Energia EM) + (Energia Meccanica) \end{center}
			L'energia meccanica prodotta non rappresenta tuttavia l'energia meccanica realmente disponibile.
			In termini di potenza, il bilancio energetico può essere riscritto come :
			\[ P_e = P_d + \frac{d}{dt} (E_{EM} + E_K) + P_m \] dove $P_e$ e' la potenza elettrica fornita al sistema dai 
			generatori elettrici, $P_d$ la potenza dissipata, $E_{EM}$ l'energia elettromagnetica accumulata nel dispositivo,
			$E_K$ l'energia meccanica accumulata nel sistema e $P_m$ la potenza meccanica fornita dal dispositivo.
			Durante una trasformazione infinitesima l'equazione precedente può essere rielaborata introducendo
			le seguenti quantità:
			\begin{enumerate}
				\item $dL_e = P_e = \sum_m vi \cdot dt$ , ovvero il lavoro fornito dai generatori elettrici al sistema attraverso
					  gli $m$ morsetti che entrano in $S$. La sommatoria e' estesa a tutti i morsetti
				\item $dL_m = \sum_a \overline{F} \cdot \overline{dS} = P_m \cdot dt$ e' l'energia fornita dal sistema 
					  sotto forma di lavoro per vincere le forze meccaniche esterne. $\overline{F}$ e' la generica forza trasmessa
					  dal sistema attraverso il confine e $\overline{dS}$ lo spostamento del suo punto di applicazione.
				\item $dQ = dE_d = P_d \cdot dt$ e' il calore ceduto dal sistema all'esterno.
				\item $dE_c = \frac{d}{dt} (E_{EM} + E_K ) dt = d(E_{EM} + dE_K)$ e' l'incremento di energia elettromagnetica 
					  e meccanica del sistema. 
			\end{enumerate}
			Il bilancio energetico esteso al sistema interno ad $S$ e' dunque:
			\[ dL_e = dE_d + dE_c + dL_m = dQ _ dE_c + dL_m \]
		\subsection{Mutua Induzione}
			La mutua induttanza ( o mutua induzione) e' l'induttanza fra due circuiti elettricamente separati, quando il 
			campo magnetico generato da uno esercita una forza elettromotrice sull'altro, e viceversa. La forza elettromotrice
			indotta nel caso di mutua induttanza si scrive:
			\[ f = - \frac{d \Phi_B}{dt} = - M \frac{di}{dt} \] dove $M$ e' chiamato coefficiente di mutua induttanza ed e'
			dimensionalmente uguale ad una induttanza (Henry). Nel caso piu' generale, in cui si hanno due circuiti, ognuno
			dei quali collegato ad un generatore, bisogna tenere conto sia della mutua induttanza che dell'autoinduzione:
			\[ \begin{cases} \Phi_1 = L_1 i_1 + Mi_2 \\ \Phi_2 = Mi_1 + L_2i_2 \end{cases} \] In generale si ha questo fenomeno
			nei trasformatori.
		\subsection{Autoinduzione}
			Poiché si ha una forza contro-elettromotrice ogni volta che interviene una variazione di flusso in un campo magnetico, 
			se considerassimo un solenoide nel quale si faccia variare l'intensità della corrente, si produrrà un campo magnetico
			variabile. Man mano che l'intensità della corrente aumenta, aumenta anche il flusso del campo magnetico dalla 
			corrente stessa, quindi sul solenoide si produrrà una corrente indotta, il cui effetto e' quello di opporsi all'
			aumento della corrente inducente. Questo fenomeno prende il nome di \textbf{autoinduzione} e la forza elettromotrice
			che si genera prende il nome di \textbf{forza elettromotrice indotta}. 
			In sintesi, quando il flusso di campo magnetico concatenato con un circuito varia per effetto della variazione 
			dell'intensità della corrente del circuito stesso, la f.e.m. e' detta di autoinduzione ed ha formula 
			\[ f.e.m. indotta = -\frac{\triangle \Phi}{\triangle t} \]
		\subsection{Induttanza}
			Una corrente elettrica $i$ che scorre in un circuito elettrico produce un campo magnetico nello spazio circostante: se 
			la corrente varia nel tempo, il flusso magnetico $\Phi_B$ del campo concatenato al circuito risulta variabile, 
			determinando entro il circuito una forza elettromotrice indotta che si oppone alla variazione del flusso. 
			Il \textbf{coefficiente di autoinduzione} $L$ del circuito e' il rapporto tra il flusso del campo magnetico concatenato
			e la corrente, che nel caso di una spira e' dato da 
			\[ L = \frac{\Phi_{\overrightarrow{B}}}{i} \] L'unita' di misura dell'induttanza e' detta Henry, ed essa equivale a 
			\[ 1H = 1 Wb \cdot A^{-1} \]. In un induttore di 1 Henry una variazione di corrente di 1 ampere al secondo genera una
			forza elettromotrice di 1 volt che e' pari al flusso di 1 weber al secondo. Analogamente la \textbf{dissuadenza} di una
			semplice spira sarà pari a 
			\[ \Lambda = \frac{i}{\Phi_B} \]
			L'energia immagazzinata in un solenoide può essere espressa per mezzo della sua induttanza caratteristica $L$ e della
			corrente $i$ che scorre nelle sue spire. La sua relazione e' \[ W = \frac{1}{2} Li^2 \] dove $W$ e' 
			l'energia immagazzinata.
		\subsection{Induttore}
		\subsection{Circuito RL}
			Un circuito RL e' un circuito elettrico del primo ordine basato su una resistenza e sulla presenza di 
			un elemento dinamico, ovvero l'\textbf{induttore}
			\subsubsection{Evoluzione Libera}
				Si chiama circuito $RL$ in evoluzione il circuito composto da una resistenza e da un induttore percorso da
				corrente. Evoluzione libera significa che il circuito non ha sorgenti esterne di tensione o di corrente, e questi
				funziona con corrente alternata. L'equazione del circuito, tramite la legge di Kirchhoff, e':
				\[ i(t) + i_L(t) = 0 \rightarrow \frac{V(t)}{R} + i_L (t) = 0 \]
				La tensione segue la: \[ V(t) = L \cdot \frac{di_L (t)}{dt} = -R \cdot i_L (0) \cdot e^{-tR/L} \]
				Al rapporto $\frac{L}{R} = \tau[s]$ viene dato il nome di costante di tempo del circuito ed e' una quantità 
				caratteristica costante del circuito, da cui segue che \[ i(\tau) = \frac{1}{e} \]
			\subsubsection{Generatore di corrente costante}
				Ipotizzando che il generatore di corrente eroghi una corrente $I_0$ costante nel tempo, possiamo scrivere 
				l'equazione di Kirchhoff delle correnti:
				\[ I_0 = i(t) + i_L(t) = \frac{V(t)}{R} + i_L (t) \] dove $V(t)$ e' la tensione. La tensione segue la:
				\[ V(t) = L \cdot \frac{di_L (t)}{dt} = -R \cdot (i_L (0) - I_0) \cdot e^{-t/ \tau}  \]
				In particolare, la risposta del circuito $RL$ ad una corrente costante e' composta di due pari: il termine
				\[ i_L (0) - I_0) e^{-t/\tau} \] e detto \textbf{risposta transitoria} o transiente del circuito, mentre il
				termine $I_0$ e' la risposta permanente o a regime del circuito.
			\subsubsection{Generatore di corrente costante a tratti}
				Prendiamo un segnale a gradino del tipo
				\[ u(t) = \begin{cases} 0 \hspace{1.0cm} per \hspace{1.0cm} t < 0 \\ 1  \hspace{1.0cm} per \hspace{1.0cm}t > 0 
				\end{cases} \]
				Il calcolo della corrente ai capi di $L$ e' data per $t > 0$:
				\[ i_L (t) = I_0 ( 1 - e^{- \frac{t}{\tau}} \]
				Ovviamente invece che a $t = 0$ si può scegliere un qualsiasi istante $t_0$ con le modifiche conseguenti:
				\[ u(t - t_0) = \begin{cases} 0  \hspace{1.0cm} per \hspace{1.0cm} t < t_0 \\ 1  
				\hspace{1.0cm} per \hspace{1.0cm}t > t_0 \end{cases} \]
				Il calcolo della corrente ai capi di $L$ e' data per $t > t_0$:
				\[ i_L(t) = I_0 ) 1 - e^{-\frac{t - t_0}{\tau}} ) \cdot u(t - t_0) \]
				Si può notare che la corrente ai capi di $L$ per $t < t_0$ e' nulla, per $t > t_0$ cresce 
				esponenzialmente come se vi fosse un generatore costante:
				\[ I_0 = u(t > t_0) = 1 \cdot I_0 \]
			\subsubsection{Risposta del circuito RL all'onda quadra}
				Applicando un segnale periodico a gradino si ha un'onda quadra:
				\[ i(t) = I_0[u(t_i) - u(t_j)] \]
				La risposta del circuito $RL$ e':
				\[ i+L(t) = I_0 (1- e^{-\frac{t}{\tau}}) \]
			\subsubsection{Risposta in frequenza del circuito RL}
				Calcoliamo come si comporta un circuito RL applicando un generatore di onda sinusoidale. In questo caso
				possiamo applicare la legge di Kirchhoff per il circuito 
				\[ I_0 sin(w t) = \frac{V(t)}{R} + i_L (t) \]
				con gli stessi ragionamenti fatti all'inizio possiamo riscrivere l'equazione come:
				\[ I_0 sin(wt) =  \frac{L}{R}\frac{di_L (t)}{dt} + i_L (t) \]
				e quindi risolvere l'equazione differenziale a coefficienti costanti con termine noto:
				\[ \frac{di_L (t)}{dt} + \frac{1}{\tau}i_L (t) = \frac{I_0 sin(wt)}{\tau}\]
				nel quale $\tau =\frac{L}{R}$ e' ancora la costante di tempo del circuito . La soluzione generale e' data
				dalla somma della soluzione dell'omogenea associata 
				\[ i_L(t) = i_L(0) e^{-\frac{t}{\tau}}\] e una soluzione particolare : \[ K sin(wt + \theta \] dove $K$ e' una
				costante. Dunque, la soluzione generale e':
				\[ i+L{t} = i_L(0)e^{-\frac{t}{\tau}} + Ksin(wt + \theta) \]
				Per $t = 0$ la corrente e' nulla, mentre per $t \rightarrow \infty$ essa tende asintoticamente al valore
				massimo $i = \frac{f_g}{R}$ che e' dato dalla prima legge di Ohm
			\subsubsection{Chiusura del Circuito}
				Il circuito comprende un generatore di forza elettromotrice $f_g$, una resistenza $R$ e una induttanza $I$.
				Alla chiusura del circuito, dalla seconda legge di Kirchhoff ricaviamo l'equazione del circuito:
				\[ f_g - L\frac{di}{dt} = Ri \rightarrow Li' + Ri = f_g \]
				La soluzione dell'equazione del circuito che rispetta la condizione iniziale data e':
				\[ i(t) = \frac{f_g}{R}(1 - e^{-\frac{R}{L}t}) \]
			\subsubsection{Apertura del circuito}
				Supponiamo di disinserire il generatore. L'equazione del circuito e' semplicemente \[Li' + Ri = 0\] che ammette
				soluzione \[I_{om} = Ae^{-\frac{R}{L}t} \] Per determinare il valore della costante indeterminata $A$, si deve
				imporre la condizione che all'istante in cui l'interruttore viene aperto la corrente abbia praticamente raggiunto
				il valore  $i = \frac{f_g}{R}$. Si ottiene quindi che nel circuito circola ancora l'extracorrente di apertura:
				\[ i(t) = \frac{f_g}{R}e^{-\frac{R}{L}t}\] che tende esponenzialmente a zero quando il tempo tende all'infinito.		
			
	\newpage
	\section{Energia Magnetica}
		\subsection{Definizione}
			L'energia magnetica e' l'energia associata al campo magnetico. Insieme all'energia potenziale elettrica, essa 
			costituisce l'energia del campo elettromagnetico. L'energia potenziale di un magnete con momento magnetico $m$ in un
			campo magnetico $B$ e' definita come il lavoro della forza magnetica (momento magnetico) nel ri-allineare il momento di
			dipolo magnetico, ed e' pari a :
			\[E = -m \cdot B \] dove nel caso di una spira di vettore area $A$ percorsa da corrente $I$ si ha $m = IA$. L'energia
			immagazzinata in un induttore di induttanza $L$ percorso da corrente $I$ e' invece $LI^2 / 2$.
			Nel caso di materiali in cui la relazione tra $B$ e $H$ sia lineare, l'energia magnetica contenuta in un volume 
			$\tau$ e' data da :
			\[ U_m = \frac{1}{2} \int_{\tau} H \cdot B d\tau = \int_{\tau} d\tau \] dove $u_m = \frac{1}{2}H \cdot B$ e'
			la densità di energia magnetica. 
		\subsection{Circuito RL}
			Per ricavare l'espressione della densità di energia del campo magnetico e' possibile considerare il caso di un 
			circuito RL nel quale sia presente un solenoide infinito ideale di induttanza $L$ e un resistore di resistenza $R$.
			Per la geometria del solenoide, in cui $S$ e' la sezione e $N$ il numero di spire, si procede in questo modo:
			\[ L \frac{dI}{dt} = \frac{d\Phi}{dt} = \frac{d}{dt}(NSB) = NS \frac{dB}{dt} \] 
			L'equazione che governa il circuito e':
			\[f - L \frac{dI}{dt} =  RI \] Sostituendo in quest'ultima la prima e moltiplicando per $I dt$:
			\[ fIdt = RI^2 dt + ISN \frac{dB}{dt} dt = RI^2 dt + ISN dB \]
			Notiamo come l'energia somministrata all'induttanza in un tempo $dt$, che e' interpretata come l'energia 
			necessaria ad aumentare l'intensità del campo $dB$, e'
			\[ dE_H = INS dB = S\cdot lnI \cdot dB \] dove $l$ e' la lunghezza del solenoide e $n$ la densità di spire. Dividendo
			per il volume del solenoide $Sl$:
			\[ dp_H = \frac{dE_H}{Sl} = nIdB = HdB \] Tale relazione ha validità generale, ma per l'esatto calcolo dell'energia e'
			necessario conoscere la relazione tra $B$ e $H$, cioè la curva di isteresi. Nel caso di materiali diamagnetici
			(tipologia di magnetismo molto debole), la loro relazione vale \[ B = \mu H\] dove $\mu$ e' la permeabilità magnetica 
			del materiale. L'energia e' facilmente calcolabile tramite una espressione analoga a quella del campo elettrico:
			\[ E_H = \int_V \frac{1}{2}H\cdot B d^3 r = \int_V p_H d^3 r \] dove $p_H = \frac{1}{2} H \cdot B$ e' la
			\textbf{pressione magnetica}
		\subsection{Induttore}
			L'energia magnetica può essere definita attraverso un induttore a partire dal fenomeno dell'autoinduzione: dalla
			legge di Faraday-Neumann-Lenz deriva che se il flusso del campo magnetico concatenato con il circuito varia nel tempo si
			produce nel circuito stesso una forza elettromotrice data dalla derivata rispetto al tempo del flusso del campo di 
			induzione magnetica cambiata di segno. La corrente che attraversa il circuito per via della forza elettromotrice
			indotta e' data:
			\[ i = -\frac{1}{R}\frac{\partial \Phi}{\partial t} \]
			Viceversa tutte le volte che facciamo circolare la corrente che percorre il circuito produciamo una forza elettromotrice 
			indotta e questo e' il fenomeno dell'autoinduzione. In questo modo si vede che il flusso $\Phi(B)$ e' proporzionale
			alla corrente secondo la :
			\[ \Phi(B) = Li \] dove $L$ e' detto\textbf{coefficiente di autoinduzione} e si misura in $[ \Omega s ] = [H]$. 
			Si conclude che l'energia associata al campo magnetico generato da una corrente elettrica che circola in un 
			circuito e' data da :
			\[ E_H = \int^i _0 Li di = \frac{1}{2}Li^2 (t) \]
			Una forma alternativa dell'energia magnetica e' quella di considerare la sua densità, ovvero la pressione magnetica:
			\[ p_H = \frac{dE_H}{dr^3} = \int^i _0 H dB \] misurata in $Pa$ (Pascal)
			
		
	\newpage
	\section{Equazioni di Maxwell e Onde \\ Elettromagnetiche}
		\subsection{Equazioni di Maxwell nel vuoto}
			\subsubsection{Legge di Gauss Elettrica}
				\begin{itemize}
					\item Forma Locale $ \nabla \cdot E = \frac{p}{\epsilon_0}$
					\item Forma Globale $ \oint_{\partial v} E \cdot dS = \frac{\int_V p dV}{\epsilon_0}$
				\end{itemize}
			\subsubsection{Legge di Gauss Magnetica}
				\begin{itemize}
					\item Forma Locale $\nabla \cdot B = 0$
					\item Forma Globale $\oint_{\partial V} B \cdot dS = 0$
				\end{itemize}
			\subsubsection{Legge di Faraday}
				\begin{itemize}
					\item Forma Locale $\nabla \times E = - \frac{\partial B}{\partial t}$
					\item Forma Globale $\nabla_{\partial S} E \cdot dl = -\frac{\partial}{\partial t} \int_S B \cdot dS$
				\end{itemize}
			\subsubsection{Legge di Ampere-Maxwell}
				\begin{itemize}
					\item Forma Locale $\nabla \times B = \mu_0 J + \mu_0 \epsilon_0 \frac{\partial E}{\partial t}$
					\item Forma Globale $\oint_{partial S} B\cdot dl = \mu_0 \int_S J\cdot dS + \mu_0 
										\epsilon_0 \oint_S \frac{\partial E}{\partial t} \cdot dS $
				\end{itemize}
				
		\subsection{Equazioni di Maxwell nei materiali}
			\subsubsection{Legge di Gauss Elettrica}
				\begin{itemize}
					\item Forma Locale $\nabla \cdot D = p_f$
					\item Forma Globale $\oint_{\partial V} D \cdot dS = \int_V p_f dV$
				\end{itemize}
			\subsubsection{Legge di Gauss Magnetica}
				\begin{itemize}
					\item Forma Locale $\nabla \cdot B= 0$
					\item Forma Globale $\oint_{\partial V} B \cdot dS = 0$
				\end{itemize}
			\subsubsection{Legge di Faraday}
				\begin{itemize}
					\item Forma Locale  $\nabla \times E = -\frac{\partial B}{\partial t}$
					\item Forma Globale $\oint_{\partial S} E \cdot dl = -\frac{\partial}{\partial t} \int_S B \cdot dS$
				\end{itemize}
			\subsubsection{Legge di Ampere-Maxwell}
				\begin{itemize}
					\item Forma Locale $\nabla \times H = J_f + \frac{\partial D}{\partial t}$
					\item Forma Globale $\oint_{\partial S} H \cdot dl = \int_s J_f \cdot dS + \frac{\partial}{\partial t} \oint_S
										D \cdot dS $
				\end{itemize}
		\subsection*{Descrizione degli elementi delle Leggi di Maxwell}
		$S$ e' una superficie chiusa, $\partial S$ il suo contorno (la curva definita considerando una sezione di $S$, $V$ e' un
		volume e $\partial V$ la superficie che lo delimita. Gli integrali su $S$ e $\partial V$ definiscono il flusso delle 
		grandezze integrate, l'integrale di linea su $\partial S$ definisce una circuitazione mentre su $V$ e' un integrale 
		di volume. Il vettore $E$ e' il campo elettrico nel vuoto, $D = \epsilon_0 E + P$ e' il campo elettrico nei materiali
		(detto anche induzione elettrica) e tiene conto della polarizzazione elettrica $P$, $B$ e' il campo magnetico percepito in
		un punto, anche detto induzione magnetica, inoltre $H = B / \mu_0 - M$ e' un campo magnetico introdotto nei materiali, che
		tiene conto della polarizzazione magnetica $M$, e $p_f$ e' la densità di carica elettrica libera, ovvero la 
		densità di carica non confinata in un dielettrico. Il prodotto $J_f = p_f v$ e' il vettore densità di corrente elettrica 
		libera. I tensori $\epsilon, \mu$ sono rispettivamente la permittività elettrica e la permeabilità magnetica.
		Le relazioni sono:
		\[ D = \epsilon E = \epsilon_0 \epsilon_r E \hspace{1.0cm} H = \frac{B}{\mu} = \frac{B}{\mu_0 \mu_r} \hspace{1.0cm}
		\frac{1}{c^2} = \epsilon_0 \mu_0 \]
		dove $c$ e' la velocità della luce, $\epsilon_r, \mu_r$ sono dette costante dielettrica relativa e permeabilità magnetica
		relativa, e sono caratteristiche del mezzo.
		
		\subsection{Corrente di spostamento}
			Si consideri il vettore induzione elettrica, definito come:
			\[ D = \epsilon_0 E + P \] dove $E$ e' il campo elettrico e $P$ la polarizzazione elettrica. 
			\textbf{La densità di corrente di spostamento} e' definita come la variazione nel tempo del vettore induzione elettrica:
			\[ J_s = \frac{\partial D}{\partial t} = \epsilon_0 \frac{\partial E}{\partial t} + \frac{\partial P}{\partial t} \]
			\textbf{La corrente di spostamento} che attraversa una data superficie $S$ e' allora definita nella forma più' generale
			come il flusso della desnita di corrente di spostamente attraverso tale superficie:
			\[ i_s = \int_S J_S \cdot dS = \epsilon_0 \int_S \frac{\partial E(t)}{\partial t} \cdot dS \]
		\subsection{Legge di Ampere-Maxwell}
			Un campo elettrico variabile nel tempo e' sperimentalmente sorgente di un campo magnetico, rendendo necessaria una
			 estensione della legge di Ampere. Inserendo la prima legge di Maxwell nell'equazione di continuità si ottiene
			\[ \nabla \cdot J + \frac{\partial p}{\partial t} = \nabla \cdot ( j + \epsilon \frac{\partial E}{\partial t} \]
			Inserendo la densità di corrente generalizzata nella legge di Ampere nel vuoto:
			\[ \nabla \times B = \mu_0 (J + \epsilon_0 \frac{\partial E}{\partial t} \]
			si ottiene la legge di Ampere-Maxwell nel vuoto.

		\subsection{Definizione di Onda Elettromagnetica}
			Si supponga di trovarsi in un dielettrico omogeneo ed isotropo, elettricamente neuotr e perfetto, privi di cariche 
			libere localizzate, sorgenti del campo elettromagnetico. Le equazione che descrivono la propagaione del campo
			sono le equazioni delle onde per il campo elettrico e magnetico, due equazioni differenziali alle derivate 
			parziali vettoriali:
			\[ \nabla^2 E - \epsilon \mu \frac{\partial^2 E}{\partial t^2} = 0 \hspace{1.5cm} 
			\nabla^2 B - \epsilon \mu \frac{\partial^2 B}{\partial t^2} = 0 \]
			La soluzione generale dell'equazione delle onde in una dimensione e' una onda:
			\[ u(x,t) = F(x - vt) + G(x + vt) \] che si propaga con velocità $v$ costante :
			\[ v = \frac{1}{\sqrt{\epsilon \mu}} \] Nel vuoto $v$ diventa la velocità della luce.
			Una rappresentazione compatta della equazione dell'onda e' ottenuta tramite l'uso dell'operatore di d'Alembert, definito
			come 
			\[ \square = \frac{\partial^2}{\partial x^2} + \frac{\partial^2}{\partial y^2} + \frac{\partial^2}{\partial z^2}  
			+ \frac{1}{v^2} \frac{\partial^2}{\partial t^2} = \nabla^2 - \frac{1}{c^2}\frac{\partial^2}{\partial t^2}\]
			e in questo modo le equazioni delle onde si scrivono:
			\[ \square E = 0 \hspace{1.5cm} \square B = 0 \]
		\subsection{Proprietà di un'onda elettromagnetica}
			Le equazioni di Maxwell forniscono diverse informazioni riguardanti la propagazione delle onde elettromagnetiche. 
			Si consideri un generico campo:
			\[ E = E_0 f (\widehat{k} \cdot x - c_0 t)\] dove $E_0$ e' l'ampiezza costante, $f$ e' una funzione differenziabile al
			secondo ordine, $\widehat{k}$ e' il versore della direzione di propagazione e $x$ la posizione. 
			Tramite Maxwell possiamo scrivere che 
			\[ \nabla \cdot E = \widehat{k} \cdot E_0 f' (\widehat{k} \cdot x - c_0 t) = 0 \hspace{1.0cm} E \cdot\widehat{k} = 0\]
			\[ \nabla \times E = \widehat{k} \times E_0 f' (\widehat{k} \cdot x - c_0 t) = -\frac{\partial B}{\partial t}
			\hspace{1.0cm} B = \frac{1}{c_0}\widehat{k} \times E\]
			Dalle equazioni di Maxwell si evince che in una onda elettromagnetica i campi sono ortogonali fra loro e ortogonali 
			alla direzione di propagazione, che le loro ampiezze sono proporzionali e che la costante di tale proporzionalità e'
			la velocità di propagazione, che dipende dalle caratteristiche del mezzo in cui si propaga.
		\subsection{Energia e Vettore di Poynting}
			Ogni onda elettromagnetica e' in grado di trasferire energia tra due punti dello spazio. Si consideri il caso di una
			onda piana, e si prenda un volume arbitrario $\tau$ contenente un campo elettromagnetico. Al suo interno la densità
			di energia elettrica vale :
			\[ u_e = \frac{1}{2} E \cdot D \] mentre la densità di energia magnetica vale : \[u_m = \frac{1}{2} B \cdot H \].
			L'energia totale all'interno del volume sarà quindi 
			\[ U = \int_V u_E dV + \int_V u_m dV = \int_V (\frac{E \cdot D}{2} + \frac{H \cdot B}{2}) dV \]
			Derivando questa equazione e sfurttando le relazioni tra gli operatori rotore e divergenza si ottiene:
			\[ \frac{\partial U}{\partial t} = - \int_S (E \times H) dS - \int_V E \cdot J dV \] Il termine $P = E \times H$ e' 
			\textbf{il vettore di Poynting}, mentre il secondo integrale al secondo membro rappresenta il contributo dell'energia
			del campo elettrico per la presenza della carica contenuta nel volume $V$.

		
\end{document}